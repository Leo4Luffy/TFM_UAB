\usepackage{caption}
\usepackage[pass]{geometry} % load the package, but none of the default settings
\usepackage{pdflscape} % landscape pages and rotation
\usepackage{bookmark} % section links in pdf
\usepackage{enumitem} % formatting lists
\usepackage[dvipsnames]{xcolor} % prevent error in pdfpages
\usepackage{pdfpages} % for adding the book cover

\let\oldhref\href % save command before redefininig, so we can turn it off again
\renewcommand{\href}[2]{#2\footnote{\url{#1}}} % same as "link-as-notes: true"

\newcommand{\blandscape}{\begin{landscape}}
\newcommand{\elandscape}{\end{landscape}}
\AtBeginDocument{\let\maketitle\relax} % don't make automatic title page as first page

\newcommand{\CoverName}{cover} % to set page numbers of cover pages to "cover"

% change toc depth (to remove subsections in Dutch Summary from toc)
\newcommand{\changelocaltocdepth}[1]{%
  \addtocontents{toc}{\protect\setcounter{tocdepth}{#1}}%
  \setcounter{tocdepth}{#1}%
}

% From https://tex.stackexchange.com/questions/32547/how-to-measure-the-width-of-a-longtable-dynamically-and-use-this-width-in-footer
% papaja requires the \getlongtablewidth command when using the longtable=TRUE option with apa_table
\makeatletter
\newcommand\LastLTentrywidth{1em}
\newlength\longtablewidth
\setlength{\longtablewidth}{1in}
\newcommand{\getlongtablewidth}{\begingroup \ifcsname LT@\roman{LT@tables}\endcsname \global\longtablewidth=0pt \renewcommand{\LT@entry}[2]{\global\advance\longtablewidth by ##2\relax\gdef\LastLTentrywidth{##2}}\@nameuse{LT@\roman{LT@tables}} \fi \endgroup}
\makeatother

%% Typefaces
%\usepackage{fontspec}
%\setmainfont{Minion Pro}
%\setsansfont[Ligatures=TeX]{Helvetica}
%\setmathsfont(Digits,Greek,Latin)[Numbers={Proportional}]{Minion Pro}
%\setmathrm{Minion Pro}

%% Page layout

% The length of the lowercase alphabet in 11 pt Minion Pro is 127.80293pt (116.7151 pt for 10pt).
% According to equation 2.1 of the memoir manual, the optimal width of the typeblock (66 characters) would then be 294.38358306 pt, or 103.9 mm
% According to table 2.2 of the memoir manual, the typeblock should be between 22 pica's (= 93.13 mm), which would be 59 characters wide, or 26 pica's a little bit more than 22 pica's (= 110.1 mm), which would be 70 characters wide. 

%\setstocksize{240mm}{170mm} % adjusted B5, with no bleed on each side for now
%\settrimmedsize{240mm}{170mm}{*} % adjusted B5 (standard thesis size)
%\setpageml{\paperheight}{\paperwidth}{*} % center the adjusted B5 page to the middle left (so the right, bottom and top will be trimmed)
%\settypeblocksize{*}{105mm}{1.618} % typeblock of 105 mm wide (little wider than optimal, to save paper). The golden ratio is a good rule to set the height, which amounts to about 105*1.618 = 170. The actual height of the text block will differ slightly, because it has to fit an integer number of lines.
%\setlrmargins{*}{40mm}{*} % leaves 170-105 = 65 mm for margins. Set the foredge margin so its relation to the spine margin is about the golden ratio as well (65/1.618 is 40.2). A foreedge of twice the spine is also common, but I think this makes the foreedge a bit too big. The spine is then 65-40 = 25 mm.
%\setulmargins{25mm}{*}{*} % the top margin is often 1/9 of the page height, or 1/9 * 240 mm = 27 mm. Often the top margin is also the same as the spine. Both of these rules almost converge here. 
% This automatically determines the bottom margin at 240 - 25 - 170 = 45. This is bigger than the top margin, which is good, as this is where you hold the book (often the bottom margin is even twice as big as the top).
%\setheadfoot{\onelineskip}{2\onelineskip} % defaults from memoir manual
%\setheaderspaces{*}{2\onelineskip}{*} % defaults from memoir manual
%\setmarginnotes{5mm}{15mm}{\onelineskip} % too narrow (15mm, with 5 mm separation from text) for actual margin notes; but some chapter / pagestyles (e.g. companion) run off the page if this is not set.
%\settypeoutlayoutunit{mm} % use mm for printing to the log
%\checkandfixthelayout

\usepackage{color}
\usepackage{framed}
\setlength{\fboxsep}{.8em}

\newenvironment{blackbox}{
  %\definecolor{shadecolor}{rgb}{0, 0, 0}  % black
  \color{black}
  \begin{shaded}}
 {\end{shaded}}
 
\newenvironment{infobox}[1]
  {
  \begin{itemize}
  \renewcommand{\labelitemi}{
    \raisebox{-.7\height}[0pt][0pt]{
      {\setkeys{Gin}{width=3em,keepaspectratio}
        \includegraphics{figures/exam.png}}
    }
  }
  \setlength{\fboxsep}{1em}
  \begin{blackbox}
  \item
  }
  {
  \end{blackbox}
  \end{itemize}
  }

\newenvironment{cols}[1][]{}{}

\newenvironment{col}[1]{\begin{minipage}{#1}\ignorespaces}{%
\end{minipage}
\ifhmode\unskip\fi
\aftergroup\useignorespacesandallpars}

\def\useignorespacesandallpars#1\ignorespaces\fi{%
#1\fi\ignorespacesandallpars}

\makeatletter
\def\ignorespacesandallpars{%
  \@ifnextchar\par
    {\expandafter\ignorespacesandallpars\@gobble}%
    {}%
}
\makeatother
