% This is the default LaTeX template (default-1.17.0.2.tex) from the RMarkdown package, from:
% https://github.com/rstudio/rmarkdown/blob/master/inst/rmd/latex/default-1.17.0.2.tex
%
% New additions to the template are marked with "LCR"

%\documentclass[12pt,spanish,]{book} %LCR
 % if not, force the oneside and a4paper options, which seem to be the only reasonable defaults
\documentclass[12pt,spanish,a4paper,oneside,]{book} %LCR

\usepackage{lmodern}
\usepackage{amssymb,amsmath}
\usepackage{ifxetex,ifluatex}
\usepackage{fixltx2e} % provides \textsubscript
\ifnum 0\ifxetex 1\fi\ifluatex 1\fi=0 % if pdftex
  \usepackage[T1]{fontenc}
  \usepackage[utf8]{inputenc}
\else % if luatex or xelatex
  \ifxetex
    \usepackage{mathspec}
  \else
    \usepackage{fontspec}
  \fi
  \defaultfontfeatures{Ligatures=TeX,Scale=MatchLowercase}
\fi
% use upquote if available, for straight quotes in verbatim environments
\IfFileExists{upquote.sty}{\usepackage{upquote}}{}
% use microtype if available
\IfFileExists{microtype.sty}{%
\usepackage{microtype}
\UseMicrotypeSet[protrusion]{basicmath} % disable protrusion for tt fonts
}{}

 %LCR
\usepackage{hyperref}
\hypersetup{unicode=true,
            pdftitle={Colocar el titulo del TFM aquí},
            pdfauthor={Jorge Leonardo López Martínez},
            pdfborder={0 0 0},
            breaklinks=true}
\urlstyle{same}  % don't use monospace font for urls
\ifnum 0\ifxetex 1\fi\ifluatex 1\fi=0 % if pdftex
  \usepackage[shorthands=off,dutch,main=spanish]{babel}
  \newcommand{\textdutch}[2][]{\foreignlanguage{dutch}{#2}}
  \newenvironment{dutch}[2][]{\begin{otherlanguage}{dutch}}{\end{otherlanguage}}
\else
\usepackage{polyglossia}
  \setmainlanguage{spanish}
  % Tabla en lugar de cuadro
  \gappto\captionsspanish{\renewcommand{\tablename}{Tabla}  
          \renewcommand{\listtablename}{Índice de tablas}}
\else
  \usepackage[spanish,es-tabla]{babel}
\fi
\usepackage{longtable,booktabs}
\usepackage{graphicx,grffile}
\makeatletter
\def\maxwidth{\ifdim\Gin@nat@width>\linewidth\linewidth\else\Gin@nat@width\fi}
\def\maxheight{\ifdim\Gin@nat@height>\textheight\textheight\else\Gin@nat@height\fi}
\makeatother
% Scale images if necessary, so that they will not overflow the page
% margins by default, and it is still possible to overwrite the defaults
% using explicit options in \includegraphics[width, height, ...]{}
\setkeys{Gin}{width=\maxwidth,height=\maxheight,keepaspectratio}
% Make links footnotes instead of hotlinks:
\renewcommand{\href}[2]{#2\footnote{\url{#1}}}
\setlength{\emergencystretch}{3em}  % prevent overfull lines
\providecommand{\tightlist}{%
  \setlength{\itemsep}{0pt}\setlength{\parskip}{0pt}}
\setcounter{secnumdepth}{5}
% Redefines (sub)paragraphs to behave more like sections
\ifx\paragraph\undefined\else
\let\oldparagraph\paragraph
\renewcommand{\paragraph}[1]{\oldparagraph{#1}\mbox{}}
\fi
\ifx\subparagraph\undefined\else
\let\oldsubparagraph\subparagraph
\renewcommand{\subparagraph}[1]{\oldsubparagraph{#1}\mbox{}}
\fi

% LCR fix for new required cslreferences environment in pandoc
% from https://github.com/rstudio/rticles/pull/335/commits/a9937b6
% originally proposed by LS: https://github.com/LDSamson/amsterdown/commit/4d9841e
% Pandoc citation processing

%%% Use protect on footnotes to avoid problems with footnotes in titles
\let\rmarkdownfootnote\footnote%
\def\footnote{\protect\rmarkdownfootnote}

%%% This fixes a TexLive 2019 change that broke pandoc template. Will also be fixed in pandoc 2.8 %LCR
% https://github.com/jgm/pandoc/issues/5801
\renewcommand{\linethickness}{0.05em}


%%%%%%%%%%%%% BEGIN DOCUMENT %%%%%%%%%%%%%
\begin{document}

%% Page I: the half-title / "Franse pagina" %LCR
\frontmatter
\thispagestyle{empty}
\def\drop{.1\textheight}

\vspace*{\drop}
\begin{center}
\Huge \textsc{Colocar el titulo del TFM aquí}
\end{center}

%% Page II: Colophon %LCR
\clearpage
\thispagestyle{empty}
\vspace*{\fill}
\begingroup % to change formatting only temporarily
\small
\setlength{\parskip}{\baselineskip} % add space between paragraphs
\setlength\parindent{0pt} % no indents

Esta tesis se escribio usando los paquetes de R (R) Markdown, \LaTeX\ , \verb+bookdown+  y \verb+amsterdown+.

\vspace{\baselineskip}
\includegraphics{_bookdown_files/CC-BY.png} \newline
Una versión en línea de esta tesis esta disponible en 
\url{https://github.com/Leo4Luffy/TFM_UAB},
bajo la licencia Creative Commons Attribution-NonCommercial-ShareAlike 4.0 International License.
\endgroup

%% Page III: `Title page' mandated by University of Amsterdam %LCR
\clearpage
\thispagestyle{empty}
\begin{center}
\includegraphics[width=44mm]{_bookdown_files/logo_uab.png} \includegraphics[width=44mm]{_bookdown_files/logo_upv.jpg} \includegraphics[width=44mm]{_bookdown_files/logo_ciheam.jpg} \newline
\vspace{\baselineskip}
\Large\textit{Máster interuniversitario en mejora genética y biotecnología de la reproducción}\par
\vspace{\baselineskip}
\Huge\textbf{Colocar el titulo del TFM aquí}\par
\vspace{\baselineskip}
\linespread{1.3}{\normalsize Tesis académica para obtener\\
el grado de Máster en Mejora Genética y\\
Biotecnología de la Reproducción bajo la\\
dirección del prof. dr. Miguel Pérez Enciso\\ % make sure this is the current rector magnificus
\mbox{ante una comisión constituida por la Junta del Máster,}\\
para ser defendido en publico el\\
Colocar aquí la fecha de la defensa, a las colocar la hora aquí  \\ }\par %
\vspace{\baselineskip}
{\Large Jorge Leonardo López Martínez}\par
\vspace{\baselineskip}
\hfill\includegraphics[width=44mm]{_bookdown_files/logo_crag.png}\hspace*{\fill} \newline
\end{center}

%% Page IV: info on thesis committee %LCR
\clearpage
\thispagestyle{empty}
\noindent\textbf{Dirección:}\\
\\
\noindent\begin{tabular}{@{}lll}

Director:
&  prof. dr. M. Pérez-Enciso & Centre for Research in Agricultural Genomics\\

\\
\end{tabular}\\

%%%%%%%%%%%%%%%%%%


{
\setcounter{tocdepth}{1}
\tableofcontents
}
\mainmatter
\hypertarget{revisiuxf3n-de-literatura}{%
\chapter{Revisión de literatura}\label{revisiuxf3n-de-literatura}}

\hypertarget{section}{%
\section{}\label{section}}

\hypertarget{titulo}{%
\chapter{Titulo}\label{titulo}}

\textbf{Resumen}

\noindent 
Insert abstract.

\begin{center}\rule{0.5\linewidth}{0.5pt}\end{center}

\vspace*{\fill}

\noindent
\emph{Possibly insert citation here.}
\newpage

\hypertarget{intro2}{%
\section{Introducción}\label{intro2}}

La teoría de la genética en el estudio de caracteres cuantitativos se estableció hace más de un siglo cuando Ronald Fisher presentó un documento (Fisher \protect\hyperlink{ref-cite:1}{1918}) donde dio a conocer el desarrollo de la teoría del modelo infinitesimal, permitiendo con ello unificar dos de las escuelas de pensamiento que para ese entonces estaban en constante debate: la escuela de pensamiento Mendeliano, cuyo objetivo consistía en localizar y caracterizar factores de herencia, y la escuela de pensamiento biométrico, cuyo origen se remonta a Galton quien buscaba aplicar modelos biométricos con el fin de estudiar las relaciones entre parientes (Nelson, Pettersson, y Carlborg \protect\hyperlink{ref-cite:2}{2012}; Blasco y Toro \protect\hyperlink{ref-cite:3}{2014}).

La teoría del modelo infinitesimal desarrollado por Fisher establece que la varianza genética de un carácter esta determinado por un gran número de factores Mendelianos, cada uno de los cuales tiene una pequeña contribución aditiva al fenotipo de dicho carácter (Nelson, Pettersson, y Carlborg \protect\hyperlink{ref-cite:2}{2012}; Turelli \protect\hyperlink{ref-cite:9}{2017}). Naturalmente, los modelos usados en estudios de mejoramiento genético han sido concebidos en base a esta teoría (Villemereuil et~al. \protect\hyperlink{ref-cite:4}{2016}; Pérez-Enciso \protect\hyperlink{ref-cite:5}{2017}), siendo ejemplo de ello el mejor predictor lineal insesgado (BLUP, sus siglas en ingles) y el mejor predictor lineal insesgado genómico (GBLUP, sus siglas en ingles).

En las ciencias animales, el valor de cría estimado (EBV, sus siglas en ingles) se predice en función de un conjunto de modelos que relacionan el fenotipo de una población con la información de su pedigrí, mediante el uso del BLUP. No obstante, este método no es factible para poblaciones sin información de pedigrí o con una estructura poblacional compleja, como suele ser el caso de las plantas (Nakaya y Isobe \protect\hyperlink{ref-cite:6}{2012}; Tong y Nikoloski \protect\hyperlink{ref-cite:7}{2021}). Para el año 2001, Meuwissen et al.~propusieron un método innovador para predecir los valores de cría basado en marcadores de ADN en todo el genoma (GEBV, sus siglas en ingles), denominándose tiempo después como de selección genómica (Blasco y Toro \protect\hyperlink{ref-cite:3}{2014}; Nakaya y Isobe \protect\hyperlink{ref-cite:6}{2012}), el cual permitió también superar las limitaciones que suponia el uso del BLUP para predecir los valores de cría en plantas.

Hoy en día, la selección genómica se considera como un método potencial para el mejoramiento genético en plantas (Nakaya y Isobe \protect\hyperlink{ref-cite:6}{2012}), ya que los ciclos reproductivos en estas suelen ser prolongados, por lo cual con el uso de la selección genómica es posible acelerar dichos ciclos reproductivos con el beneficio adicional de mejorar la tasa de ganancia genética anual por unidad de tiempo y costo (Desta y Ortiz \protect\hyperlink{ref-cite:10}{2014}; Jurcic et~al. \protect\hyperlink{ref-cite:11}{2021}). Además, los datos sobre marcadores de ADN en todo el genoma están cada vez más disponibles para cultivos de relevancia agronómica (Tong y Nikoloski \protect\hyperlink{ref-cite:7}{2021}).

El GBLUP es uno de los métodos más comunes de selección genómica (Jurcic et~al. \protect\hyperlink{ref-cite:11}{2021}). De hecho, es el método más popular debido a su simplicidad al sustituir la matriz de relación de parentesco basado en pedigríes (Wright \protect\hyperlink{ref-cite:12}{1922}) por una matriz de relación basada en marcadores de ADN (Hayes, Visscher, y Goddard \protect\hyperlink{ref-cite:13}{2009}). Si bien (FALTARIA COLOCAR AQUI LO QUE CREO QUE DIJO LEGARRA DE QUE LA SELECCIÓN GENÓMICA ES UN BLUP MODIFICADO), la selección genómica predice el GEBV en comparación al EBV con mayor precisión, debido a que estima mejor las relaciones entre individuos (Misztal, Aggrrey, y Muir \protect\hyperlink{ref-cite:14}{2012}), por lo cual la matriz de las relaciones genómicas puede verse como un estimador mejorado de relaciones basadas en marcadores en lugar de pedigríes (Legarra et~al. \protect\hyperlink{ref-cite:15}{2014}).

\hypertarget{methods2}{%
\section{Metodos}\label{methods2}}

\hypertarget{results2}{%
\section{Resultados}\label{results2}}

\hypertarget{discussion2}{%
\section{Discusión}\label{discussion2}}

\hypertarget{titulo-1}{%
\chapter{Titulo}\label{titulo-1}}

\textbf{Resumen}

\noindent 
Insert abstract.

\begin{center}\rule{0.5\linewidth}{0.5pt}\end{center}

\vspace*{\fill}

\noindent
\emph{Possibly insert citation here.}
\newpage

\hypertarget{intro3}{%
\section{Introducción}\label{intro3}}

\hypertarget{methods3}{%
\section{Metodos}\label{methods3}}

\hypertarget{results3}{%
\section{Resultados}\label{results3}}

\hypertarget{discussion3}{%
\section{Discusión}\label{discussion3}}

\hypertarget{resumen-y-discusiuxf3n}{%
\chapter{Resumen y discusión}\label{resumen-y-discusiuxf3n}}

Here's where you would write a summary of your thesis\footnote{You can also put the summary at the end (with the Dutch summary) or even at the beginning.}, along with a general discussion.

\hypertarget{appendix-appendix}{%
\appendix}


\hypertarget{anexo-del-capitulo-2}{%
\chapter{Anexo del capitulo 2}\label{anexo-del-capitulo-2}}

What's left to say? How about a nice image then?

\includegraphics{figures/uvalogo_regular_p_en.pdf}

\hypertarget{anexo-del-capitulo-3}{%
\chapter{Anexo del capitulo 3}\label{anexo-del-capitulo-3}}

And now for some tables:

\begin{longtable}[]{@{}rrrr@{}}
\caption{\label{tab:beaver-2} Time series of the body temparature of a beaver.}\tabularnewline
\toprule
day & time & temp & activ\tabularnewline
\midrule
\endfirsthead
\toprule
day & time & temp & activ\tabularnewline
\midrule
\endhead
307 & 930 & 36.58 & 0\tabularnewline
307 & 940 & 36.73 & 0\tabularnewline
307 & 950 & 36.93 & 0\tabularnewline
307 & 1000 & 37.15 & 0\tabularnewline
307 & 1010 & 37.23 & 0\tabularnewline
307 & 1020 & 37.24 & 0\tabularnewline
307 & 1030 & 37.24 & 0\tabularnewline
307 & 1040 & 36.90 & 0\tabularnewline
307 & 1050 & 36.95 & 0\tabularnewline
307 & 1100 & 36.89 & 0\tabularnewline
307 & 1110 & 36.95 & 0\tabularnewline
307 & 1120 & 37.00 & 0\tabularnewline
307 & 1130 & 36.90 & 0\tabularnewline
307 & 1140 & 36.99 & 0\tabularnewline
307 & 1150 & 36.99 & 0\tabularnewline
307 & 1200 & 37.01 & 0\tabularnewline
\bottomrule
\end{longtable}

\begin{table}

\caption{\label{tab:beaver-1}This is another beaver. Seems to be running slightly colder}
\centering
\begin{tabular}[t]{rrrr}
\toprule
day & time & temp & activ\\
\midrule
346 & 840 & 36.33 & 0\\
346 & 850 & 36.34 & 0\\
346 & 900 & 36.35 & 0\\
346 & 910 & 36.42 & 0\\
346 & 920 & 36.55 & 0\\
\addlinespace
346 & 930 & 36.69 & 0\\
346 & 940 & 36.71 & 0\\
346 & 950 & 36.75 & 0\\
346 & 1000 & 36.81 & 0\\
346 & 1010 & 36.88 & 0\\
\addlinespace
346 & 1020 & 36.89 & 0\\
346 & 1030 & 36.91 & 0\\
346 & 1040 & 36.85 & 0\\
346 & 1050 & 36.89 & 0\\
346 & 1100 & 36.89 & 0\\
\addlinespace
346 & 1110 & 36.67 & 0\\
\bottomrule
\end{tabular}
\end{table}

The average body temperature of the 2nd beaver (Table \ref{tab:beaver-1}) is 36.7 (\emph{SD} = 0.22).

\backmatter

\hypertarget{bibliografuxeda}{%
\chapter*{Bibliografía}\label{bibliografuxeda}}
\addcontentsline{toc}{chapter}{Bibliografía}

\markboth{\MakeUppercase{Bibliography}}{} % have to explicitly state what to put in the heading (bug in bookdown?)
%format the references so they have a hanging indent. Remove these (and the \endgroup command) if you want regular indentation.
\begingroup
\hspace{\parindent}
\setlength{\parindent}{-0.25in}
\setlength{\leftskip}{0.25in}
\setlength{\parskip}{0pt}

\hypertarget{refs}{}
\leavevmode\hypertarget{ref-cite:3}{}%
Blasco, A., y M. A. Toro. 2014. «A short critical history of the application of genomics to animal breeding». \emph{Livestock Science} 166: 4-9.

\leavevmode\hypertarget{ref-cite:10}{}%
Desta, Z. A., y R. Ortiz. 2014. «Genomic selection: genome-wide prediction in plant improvement». \emph{Trends in Plant Science} 19 (9): 592-601.

\leavevmode\hypertarget{ref-cite:1}{}%
Fisher, R. A. 1918. «The correlaction between relatives under the supposition of Mendelian inheritance». \emph{Transactions of the Royal Society of Edinburgh} 52: 399-433.

\leavevmode\hypertarget{ref-cite:13}{}%
Hayes, B. J., P. M. Visscher, y M. E. Goddard. 2009. «Increased accuracy of artificial selection by using the realized relationship matrix». \emph{Genetics Research} 91: 47-60.

\leavevmode\hypertarget{ref-cite:11}{}%
Jurcic, E. J., P. V. Villalba, P. S. Pathauer, D. A. Palazzini, G. P. J. Oberschelp, L. Harrand, M. N. Garcia, et~al. 2021. «Genomic selection: genome-wide prediction in plant improvement». \emph{Trends in Plant Science} 127: 176-89.

\leavevmode\hypertarget{ref-cite:15}{}%
Legarra, A., O. F. Christensen, I. Aguilar, y I. Misztal. 2014. «Single Step, a general approach for genomic selection». \emph{Livestock Science}. \url{https://doi.org/http://dx.doi.org/10.1016/j.livsci.2014.04.029}.

\leavevmode\hypertarget{ref-cite:14}{}%
Misztal, I., S. E. Aggrrey, y W. M. Muir. 2012. «Experiences with a single-step genome evaluation». \emph{Poultry Science} 92: 2530-4.

\leavevmode\hypertarget{ref-cite:6}{}%
Nakaya, A., y S. N. Isobe. 2012. «Will genomic selection be a practical method for plant breeding?» \emph{Annals of Botany} 110: 1303-16.

\leavevmode\hypertarget{ref-cite:2}{}%
Nelson, R. M., M. E. Pettersson, y Ö. Carlborg. 2012. «A century after Fisher: time for a new paradigm in quantitative genetics». \emph{Trends in Genetics} 29 (9): 669-76.

\leavevmode\hypertarget{ref-cite:5}{}%
Pérez-Enciso, M. 2017. «Animal breeding learning from machine learning». \emph{Journal of Animal Breeding and Genetics} 134: 85-86.

\leavevmode\hypertarget{ref-cite:7}{}%
Tong, H., y Z. Nikoloski. 2021. «Machine learning approaches for crop improvement: leveraging phenotypic and genotypic big data». \emph{Journal of Plant Physiology} 257: 153354. \url{https://doi.org/10.1016/j.jplph.2020.153354}.

\leavevmode\hypertarget{ref-cite:9}{}%
Turelli, M. 2017. «Prediction of Total Genetic Value Using Genome-Wide Dense Marker Maps». \emph{Theoretical Population Biology} 118: 46-49.

\leavevmode\hypertarget{ref-cite:4}{}%
Villemereuil, P. de, H. Schielzeth, S. Nakagawa, y M. Morrissey. 2016. «General methods for evolutionary quantitative genetic inference from generalized mixed models». \emph{Genetics} 204: 1281-94.

\leavevmode\hypertarget{ref-cite:12}{}%
Wright, S. 1922. «Coefficients of inbreeding and relationship». \emph{The American Naturalist} 56: 330-38.

\endgroup

\hypertarget{agradecimientos}{%
\chapter*{Agradecimientos}\label{agradecimientos}}
\addcontentsline{toc}{chapter}{Agradecimientos}

\chaptermark{Acknowledgments}

This section is optional, but theses typically include acknowledgments (\textdutch{\emph{dankwoord}} in Dutch) at the end. You may want to mix languages to thank people in their native tongue (though most Dutch speakers write it entirely in Dutch). But the standard language of the thesis template is English. You can switch temporarily by wrapping the text in language tags like so: \texttt{{[}Your\ Dutch\ text\ here{]}\{lang=nl\}}. This is important for things like hyphenation to work properly.

\hypertarget{varios}{%
\chapter*{Varios}\label{varios}}
\addcontentsline{toc}{chapter}{Varios}

\chaptermark{Miscellaneous}

Sometimes more chapters are added to the back matter, for instance:

\begin{itemize}
\tightlist
\item
  A short CV of the author
\item
  A list of publications; (including) work that did not make it into the thesis (but that you've co-authored, for instance)
\item
  A list of contributions to the chapters, in case they are based on multi-author papers
\end{itemize}

I've included the latter two in \href{https://lcreteig.github.io/thesis/}{my PhD thesis}, if you're looking for some inspiration.

\backmatter

\end{document}
