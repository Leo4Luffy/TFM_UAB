% This is the default LaTeX template (default-1.17.0.2.tex) from the RMarkdown package, from:
% https://github.com/rstudio/rmarkdown/blob/master/inst/rmd/latex/default-1.17.0.2.tex
%
% New additions to the template are marked with "LCR"

%\documentclass[11pt,spanish,]{book} %LCR
 % if not, force the oneside and a4paper options, which seem to be the only reasonable defaults
\documentclass[11pt,spanish,a4paper,oneside,]{book} %LCR

\usepackage{lmodern}
\usepackage{amssymb,amsmath}
\usepackage{ifxetex,ifluatex}
\usepackage{fixltx2e} % provides \textsubscript
\ifnum 0\ifxetex 1\fi\ifluatex 1\fi=0 % if pdftex
  \usepackage[T1]{fontenc}
  \usepackage[utf8]{inputenc}
\else % if luatex or xelatex
  \ifxetex
    \usepackage{mathspec}
  \else
    \usepackage{fontspec}
  \fi
  \defaultfontfeatures{Ligatures=TeX,Scale=MatchLowercase}
\fi
% use upquote if available, for straight quotes in verbatim environments
\IfFileExists{upquote.sty}{\usepackage{upquote}}{}
% use microtype if available
\IfFileExists{microtype.sty}{%
\usepackage{microtype}
\UseMicrotypeSet[protrusion]{basicmath} % disable protrusion for tt fonts
}{}

 %LCR
\usepackage{hyperref}
\PassOptionsToPackage{usenames,dvipsnames}{color} % color is loaded by hyperref
\hypersetup{unicode=true,
            pdftitle={Colocar el titulo del TFM aquí},
            pdfauthor={Jorge Leonardo López Martínez},
            colorlinks=true,
            linkcolor=cyan,
            citecolor=Blue,
            urlcolor=cyan,
            breaklinks=true}
\urlstyle{same}  % don't use monospace font for urls
\ifnum 0\ifxetex 1\fi\ifluatex 1\fi=0 % if pdftex
  \usepackage[shorthands=off,main=spanish]{babel}
\else
\usepackage{polyglossia}
  \setmainlanguage{spanish}
  % Tabla en lugar de cuadro
  \gappto\captionsspanish{\renewcommand{\tablename}{Tabla}  
          \renewcommand{\listtablename}{Índice de tablas}}
\else
  \usepackage[spanish,es-tabla]{babel}
\fi
\usepackage{color}
\usepackage{fancyvrb}
\newcommand{\VerbBar}{|}
\newcommand{\VERB}{\Verb[commandchars=\\\{\}]}
\DefineVerbatimEnvironment{Highlighting}{Verbatim}{commandchars=\\\{\}}
% Add ',fontsize=\small' for more characters per line
\usepackage{framed}
\definecolor{shadecolor}{RGB}{248,248,248}
\newenvironment{Shaded}{\begin{snugshade}}{\end{snugshade}}
\newcommand{\AlertTok}[1]{\textcolor[rgb]{0.94,0.16,0.16}{#1}}
\newcommand{\AnnotationTok}[1]{\textcolor[rgb]{0.56,0.35,0.01}{\textbf{\textit{#1}}}}
\newcommand{\AttributeTok}[1]{\textcolor[rgb]{0.77,0.63,0.00}{#1}}
\newcommand{\BaseNTok}[1]{\textcolor[rgb]{0.00,0.00,0.81}{#1}}
\newcommand{\BuiltInTok}[1]{#1}
\newcommand{\CharTok}[1]{\textcolor[rgb]{0.31,0.60,0.02}{#1}}
\newcommand{\CommentTok}[1]{\textcolor[rgb]{0.56,0.35,0.01}{\textit{#1}}}
\newcommand{\CommentVarTok}[1]{\textcolor[rgb]{0.56,0.35,0.01}{\textbf{\textit{#1}}}}
\newcommand{\ConstantTok}[1]{\textcolor[rgb]{0.00,0.00,0.00}{#1}}
\newcommand{\ControlFlowTok}[1]{\textcolor[rgb]{0.13,0.29,0.53}{\textbf{#1}}}
\newcommand{\DataTypeTok}[1]{\textcolor[rgb]{0.13,0.29,0.53}{#1}}
\newcommand{\DecValTok}[1]{\textcolor[rgb]{0.00,0.00,0.81}{#1}}
\newcommand{\DocumentationTok}[1]{\textcolor[rgb]{0.56,0.35,0.01}{\textbf{\textit{#1}}}}
\newcommand{\ErrorTok}[1]{\textcolor[rgb]{0.64,0.00,0.00}{\textbf{#1}}}
\newcommand{\ExtensionTok}[1]{#1}
\newcommand{\FloatTok}[1]{\textcolor[rgb]{0.00,0.00,0.81}{#1}}
\newcommand{\FunctionTok}[1]{\textcolor[rgb]{0.00,0.00,0.00}{#1}}
\newcommand{\ImportTok}[1]{#1}
\newcommand{\InformationTok}[1]{\textcolor[rgb]{0.56,0.35,0.01}{\textbf{\textit{#1}}}}
\newcommand{\KeywordTok}[1]{\textcolor[rgb]{0.13,0.29,0.53}{\textbf{#1}}}
\newcommand{\NormalTok}[1]{#1}
\newcommand{\OperatorTok}[1]{\textcolor[rgb]{0.81,0.36,0.00}{\textbf{#1}}}
\newcommand{\OtherTok}[1]{\textcolor[rgb]{0.56,0.35,0.01}{#1}}
\newcommand{\PreprocessorTok}[1]{\textcolor[rgb]{0.56,0.35,0.01}{\textit{#1}}}
\newcommand{\RegionMarkerTok}[1]{#1}
\newcommand{\SpecialCharTok}[1]{\textcolor[rgb]{0.00,0.00,0.00}{#1}}
\newcommand{\SpecialStringTok}[1]{\textcolor[rgb]{0.31,0.60,0.02}{#1}}
\newcommand{\StringTok}[1]{\textcolor[rgb]{0.31,0.60,0.02}{#1}}
\newcommand{\VariableTok}[1]{\textcolor[rgb]{0.00,0.00,0.00}{#1}}
\newcommand{\VerbatimStringTok}[1]{\textcolor[rgb]{0.31,0.60,0.02}{#1}}
\newcommand{\WarningTok}[1]{\textcolor[rgb]{0.56,0.35,0.01}{\textbf{\textit{#1}}}}
\usepackage{longtable,booktabs}
\usepackage{graphicx,grffile}
\makeatletter
\def\maxwidth{\ifdim\Gin@nat@width>\linewidth\linewidth\else\Gin@nat@width\fi}
\def\maxheight{\ifdim\Gin@nat@height>\textheight\textheight\else\Gin@nat@height\fi}
\makeatother
% Scale images if necessary, so that they will not overflow the page
% margins by default, and it is still possible to overwrite the defaults
% using explicit options in \includegraphics[width, height, ...]{}
\setkeys{Gin}{width=\maxwidth,height=\maxheight,keepaspectratio}
% Make links footnotes instead of hotlinks:
\renewcommand{\href}[2]{#2\footnote{\url{#1}}}
\setlength{\emergencystretch}{3em}  % prevent overfull lines
\providecommand{\tightlist}{%
  \setlength{\itemsep}{0pt}\setlength{\parskip}{0pt}}
\setcounter{secnumdepth}{5}
% Redefines (sub)paragraphs to behave more like sections
\ifx\paragraph\undefined\else
\let\oldparagraph\paragraph
\renewcommand{\paragraph}[1]{\oldparagraph{#1}\mbox{}}
\fi
\ifx\subparagraph\undefined\else
\let\oldsubparagraph\subparagraph
\renewcommand{\subparagraph}[1]{\oldsubparagraph{#1}\mbox{}}
\fi

% LCR fix for new required cslreferences environment in pandoc
% from https://github.com/rstudio/rticles/pull/335/commits/a9937b6
% originally proposed by LS: https://github.com/LDSamson/amsterdown/commit/4d9841e
% Pandoc citation processing

%%% Use protect on footnotes to avoid problems with footnotes in titles
\let\rmarkdownfootnote\footnote%
\def\footnote{\protect\rmarkdownfootnote}

%%% This fixes a TexLive 2019 change that broke pandoc template. Will also be fixed in pandoc 2.8 %LCR
% https://github.com/jgm/pandoc/issues/5801
\renewcommand{\linethickness}{0.05em}

\usepackage{amsmath}
\usepackage{booktabs}
\usepackage{caption}
\usepackage{longtable}

%%%%%%%%%%%%% BEGIN DOCUMENT %%%%%%%%%%%%%
\begin{document}

%% Page I: the half-title / "Franse pagina" %LCR
\frontmatter
\thispagestyle{empty}
\def\drop{.1\textheight}

\vspace*{\drop}
\begin{center}
\Huge \textsc{Colocar el titulo del TFM aquí}
\end{center}

%% Page II: Colophon %LCR
\clearpage
\thispagestyle{empty}
\vspace*{\fill}
\begingroup % to change formatting only temporarily
\small
\setlength{\parskip}{\baselineskip} % add space between paragraphs
\setlength\parindent{0pt} % no indents

Esta tesis se escribio usando los paquetes de R (R) Markdown, \LaTeX\ , \verb+bookdown+  y \verb+amsterdown+.

\vspace{\baselineskip}
\includegraphics{_bookdown_files/CC-BY.png} \newline
Una versión en línea de esta tesis esta disponible en 
\url{https://github.com/Leo4Luffy/TFM_UAB},
bajo la licencia Creative Commons Attribution-NonCommercial-ShareAlike 4.0 International License.
\endgroup

%% Page III: `Title page' mandated by University of Amsterdam %LCR
\clearpage
\thispagestyle{empty}
\begin{center}
\includegraphics[width=41mm]{_bookdown_files/logo_uab.png} \includegraphics[width=41mm]{_bookdown_files/logo_upv.jpg} \includegraphics[width=41mm]{_bookdown_files/logo_ciheam.jpg} \newline
\vspace{\baselineskip}
\Huge\textbf{Colocar el titulo del TFM aquí}\par
\vspace{\baselineskip}
\linespread{1.3}{\normalsize Tesis académica para obtener\\
el grado de Máster en Mejora Genética y\\
Biotecnología de la Reproducción bajo la\\
dirección del prof. dr. Miguel Pérez Enciso\\ % make sure this is the current rector magnificus
\mbox{ante una comisión constituida por la Junta del Máster,}\\
para ser defendido en publico el\\
Colocar aquí la fecha de la defensa, a las colocar la hora aquí  \\ }\par %
\vspace{\baselineskip}
{\Large Jorge Leonardo López Martínez}\par
\vspace{\baselineskip}
\hfill\includegraphics[width=44mm]{_bookdown_files/logo_crag.png}\hspace*{\fill} \newline
\end{center}

%% Page IV: info on thesis committee %LCR
\clearpage
\thispagestyle{empty}
\noindent\textbf{Dirección:}\\
\\
\noindent\begin{tabular}{@{}lll}

Director:
&  prof. dr. M. Pérez-Enciso & Centre for Research in Agricultural Genomics\\

\\
\end{tabular}\\

%%%%%%%%%%%%%%%%%%


{
\hypersetup{linkcolor=black}
\setcounter{tocdepth}{1}
\tableofcontents
}
\mainmatter
\hypertarget{revisiuxf3n-de-literatura}{%
\chapter{Revisión de literatura}\label{revisiuxf3n-de-literatura}}

\hypertarget{breve-historia-hacia-la-selecciuxf3n-genuxf3mica}{%
\section{Breve historia hacia la selección genómica}\label{breve-historia-hacia-la-selecciuxf3n-genuxf3mica}}

La historia de la genética tanto cuantitativa como molecular se remonta a la contribución de muchas personas (Figura 1.1), hecho que permitió la conexión entre ambas disciplinas y el desarrollo de lo que hoy en día se conoce como selección genómica.

\begin{center}\includegraphics[width=1\linewidth]{figures/Crono} \end{center}

\begin{center}
\textbf{Figura 1.1:} Cronología de las disciplinas de la genética molecular y cuantitativa. Figura adaptada de Nelson, Pettersson, y Carlborg (\protect\hyperlink{ref-cite:2}{2012}).

\end{center}

La genética cuantitativa se formo hace más de un siglo observando el efecto de los genes en ausencia directa de datos genéticamente observables (Nelson, Pettersson, y Carlborg \protect\hyperlink{ref-cite:2}{2012}). Esta disciplina se formo gracias a los avances teóricos de Ronald Fisher quien proporcionó una teoría que hizo posible interpretar los descubrimientos de la genética biométrica dentro de los estudios de herencia Mendeliana, permitiendo con ello unificar las escuelas de pensamiento Mendeliano y biométrico de Galton. Dicha teoría (denominada como teoría del modelo infinitesimal) supuso que la herencia genética es principalmente aditiva, y que la varianza genética de un carácter esta determinado por un gran número de factores Mendelianos, cada uno de los cuales tiene una pequeña contribución al fenotipo del carácter (Nelson, Pettersson, y Carlborg \protect\hyperlink{ref-cite:2}{2012}; Turelli \protect\hyperlink{ref-cite:9}{2017}). A partir de este entonces, la genética cuantitativa fue extremadamente productiva a medida fue adhiriéndose a la teoría del modelo infinitesimal.

\hypertarget{la-selecciuxf3n-genuxf3mica}{%
\section{La selección genómica}\label{la-selecciuxf3n-genuxf3mica}}

\hypertarget{titulo}{%
\chapter{Titulo}\label{titulo}}

\textbf{Resumen}

\noindent 
Insert abstract.

\begin{center}\rule{0.5\linewidth}{0.5pt}\end{center}

\vspace*{\fill}

\noindent
\emph{Possibly insert citation here.}
\newpage

\hypertarget{intro2}{%
\section{Introducción}\label{intro2}}

La teoría de la genética en el estudio de caracteres cuantitativos se estableció hace más de un siglo cuando Ronald Fisher presentó un documento (Fisher \protect\hyperlink{ref-cite:1}{1918}) donde dio a conocer el desarrollo de la teoría del modelo infinitesimal, permitiendo con ello unificar dos de las escuelas de pensamiento que para ese entonces estaban en constante debate: la escuela de pensamiento Mendeliano, cuyo objetivo consistía en localizar y caracterizar factores de herencia, y la escuela de pensamiento biométrico, cuyo origen se remonta a Galton quien buscaba aplicar modelos biométricos con el fin de estudiar las relaciones entre parientes (Nelson, Pettersson, y Carlborg \protect\hyperlink{ref-cite:2}{2012}; Blasco y Toro \protect\hyperlink{ref-cite:3}{2014}).

La teoría del modelo infinitesimal desarrollado por Fisher establece que la varianza genética de un carácter esta determinado por un gran número de factores Mendelianos, cada uno de los cuales tiene una pequeña contribución aditiva al fenotipo de dicho carácter (Nelson, Pettersson, y Carlborg \protect\hyperlink{ref-cite:2}{2012}; Turelli \protect\hyperlink{ref-cite:9}{2017}). Naturalmente, los modelos usados en estudios de mejoramiento genético han sido concebidos en base a esta teoría (Villemereuil et~al. \protect\hyperlink{ref-cite:4}{2016}; Pérez-Enciso \protect\hyperlink{ref-cite:5}{2017}), siendo ejemplo de ello el mejor predictor lineal insesgado (BLUP) y el mejor predictor lineal insesgado genómico (GBLUP).

En las ciencias animales, el valor de cría estimado (EBV) se suele predecir en función de un conjunto de modelos que relacionan el fenotipo de una población con la información del pedigrí, mediante el uso del BLUP. No obstante, este método no es factible para poblaciones sin información de pedigrí o con una estructura poblacional compleja, como suele ser el caso de las plantas (Nakaya y Isobe \protect\hyperlink{ref-cite:6}{2012}; Tong y Nikoloski \protect\hyperlink{ref-cite:7}{2021}). Para el año 2001, Meuwissen, Hayes y Goddard propusieron un método innovador para predecir los valores de cría basado en marcadores de ADN (GEBV), denominándose tiempo después como selección genómica (Nakaya y Isobe \protect\hyperlink{ref-cite:6}{2012}; Blasco y Toro \protect\hyperlink{ref-cite:3}{2014}), el cual permitió también superar las limitaciones que suponia el uso del BLUP para predecir los valores de cría en plantas.

Hoy en día, la selección genómica se considera como un método potencial para el mejoramiento genético en plantas (Nakaya y Isobe \protect\hyperlink{ref-cite:6}{2012}), ya que sus ciclos reproductivos suelen ser prolongados, por lo cual con el uso de la selección genómica es posible acelerar dichos ciclos reproductivos con el beneficio adicional de mejorar la tasa de ganancia genética anual por unidad de tiempo y costo (Desta y Ortiz \protect\hyperlink{ref-cite:10}{2014}; Jurcic et~al. \protect\hyperlink{ref-cite:11}{2021}). Además, los datos sobre marcadores de ADN en todo el genoma están cada vez más disponibles para cultivos de relevancia agronómica (Tong y Nikoloski \protect\hyperlink{ref-cite:7}{2021}).

El GBLUP es uno de los métodos más comunes de selección genómica (Jurcic et~al. \protect\hyperlink{ref-cite:11}{2021}). De hecho, es el método más popular debido a su simplicidad al sustituir la matriz de relación de parentesco basado en pedigríes (Wright \protect\hyperlink{ref-cite:12}{1922}) por una matriz de relación basada en marcadores de ADN (Hayes, Visscher, y Goddard \protect\hyperlink{ref-cite:13}{2009}). Así mismo, el GBLUP predice con mayor precisión los GEBV en comparación a los EBV del BLUP, debido a que con el primero se estima mejor las relaciones entre individuos (Misztal, Aggrrey, y Muir \protect\hyperlink{ref-cite:14}{2012}), por lo cual la matriz de las relaciones genómicas suele verse como un estimador mejorado de las relaciones basadas en marcadores en lugar de pedigríes (Legarra et~al. \protect\hyperlink{ref-cite:15}{2014}).

En términos generales, la selección genómica es un proceso de tres pasos en el que los individuos, sobre la base de su información fenotípica y de pedigrí, son evaluados inicialmente mediante una evaluación genética tradicional por medio del BLUP, y posteriormente a partir de los fenotipos corregidos o pseudo-fenotipos resultantes de esta evaluación genética inicial, es llevado a cabo un análisis genómico de los individuos genotipados mediante el GBLUP. Por último y en base a la información generada, se calculan los GEBV por medio de un índice de selección (Legarra, Aguilar, y Misztal \protect\hyperlink{ref-cite:17}{2009}; Misztal, Legarra, y Aguilar \protect\hyperlink{ref-cite:16}{2009}; Misztal, Aggrrey, y Muir \protect\hyperlink{ref-cite:14}{2012}; Legarra et~al. \protect\hyperlink{ref-cite:15}{2014}; Misztal, Lourenco, y Legarra \protect\hyperlink{ref-cite:18}{2020}).

Como no todos los individuos pueden genotiparse, la selección genómica se lleva a cabo a partir del proceso anterior de tres pasos (Legarra, Aguilar, y Misztal \protect\hyperlink{ref-cite:17}{2009}). Sin embargo, este proceso es tendente a cometer errores (Misztal, Aggrrey, y Muir \protect\hyperlink{ref-cite:14}{2012}), además de presentar inconvenientes como son la perdida de información y la difícultad de generalizarse a caracteres múltiples y maternos (Legarra, Aguilar, y Misztal \protect\hyperlink{ref-cite:17}{2009}; Legarra et~al. \protect\hyperlink{ref-cite:15}{2014}). Conscientes de esto, Legarra, Aguilar, y Misztal (\protect\hyperlink{ref-cite:17}{2009}) simplificaron el proceso de varios pasos al desarrollar un método de selección genómica, en el que los fenotipos de los individuos genotipados y no genotipados se analizan conjuntamente para predecir sus valores de cría (Imai et~al. \protect\hyperlink{ref-cite:20}{2019}; Jurcic et~al. \protect\hyperlink{ref-cite:11}{2021}), método el cual se denomino como mejor predictor lineal insesgado genómico de un solo paso (ssGBLUP).

En el ssGBLUP se dispone de una matriz de parentesco genómica global de individuos genotipados y no genotipados, denominada como matriz de relación combinada o matriz H. Esta matriz se obtiene combinando información de la relación genómica entre individuos genotipados, e información de pedigrí entre individuos genotipados y no genotipados (Imai et~al. \protect\hyperlink{ref-cite:20}{2019}). Con ello, el proceso anterior de tres pasos tiende a simplificarse al incorporar la información genómica desde el primer paso (Legarra et~al. \protect\hyperlink{ref-cite:15}{2014}; Misztal, Legarra, y Aguilar \protect\hyperlink{ref-cite:16}{2009}), sin la necesidad del calculo posterior de fenotipos corregidos y la construcción del índice de selección mencionado previamente (Misztal, Lourenco, y Legarra \protect\hyperlink{ref-cite:18}{2020}).

Al ser una forma de BLUP en el que la matriz de relación de parentesco es sustituida por la matriz de relación combinada (Legarra, Aguilar, y Misztal \protect\hyperlink{ref-cite:17}{2009}; Legarra et~al. \protect\hyperlink{ref-cite:15}{2014}; Blasco \protect\hyperlink{ref-cite:21}{2021}), el ssGBLUP se puede adecuar con facilidad a caracteres múltiples y maternos (Blasco \protect\hyperlink{ref-cite:21}{2021}), además se adapta también a las herramientas informaticas ya desarrolladas en base al BLUP (Lourenco et~al. \protect\hyperlink{ref-cite:22}{2020}). Este hecho hace del ssGBLUP un método de uso rutinario para la evaluación genómica en animales, donde ha demostrado que produce una predicción más precisa de los valores de cría en comparación a los métodos BLUP y GBLUP antes mencionados (Misztal, Aggrrey, y Muir \protect\hyperlink{ref-cite:14}{2012}; Pérez-Rodríguez et~al. \protect\hyperlink{ref-cite:19}{2017}; Misztal, Lourenco, y Legarra \protect\hyperlink{ref-cite:18}{2020}). No obstante, el uso del ssGBLUP para la selección genómica en plantas es más reciente y escaso (Pérez-Rodríguez et~al. \protect\hyperlink{ref-cite:19}{2017}; Jurcic et~al. \protect\hyperlink{ref-cite:11}{2021}). En consecuencia, el \textbf{objetivo}

\hypertarget{methods2}{%
\section{Métodos}\label{methods2}}

\hypertarget{recurso-vegetal-y-datos-fenotuxedpicos}{%
\subsection{Recurso vegetal y datos fenotípicos}\label{recurso-vegetal-y-datos-fenotuxedpicos}}

Los conjuntos de datos se obtuvieron del \href{https://snp-seek.irri.org/index.zul;jsessionid=DD991975FDC4F320BE3C33ED056D0363}{Rice SNP-Seek Database}, el cual es un cibersitio con información sobre datos de genotipado de SNPs y de fenotipos de distintas variedades de arroz (\emph{Oryza sativa L.}). Posteriormente, dichos conjuntos de datos fueron usados por Vourlaki et~al. (\protect\hyperlink{ref-cite:26}{s.~f.}), quienes sometieron los datos de genotipado de SNPs a procedimientos de control de calidad, en los que fueron eliminados loci de SNPs con una frecuencia del alelo menor de menos de 0.01 y con una tasa de ausencia mayor a 0.01.

Mediante un análisis de componentes principales realizado sobre los datos de genotipado de SNPs (Figura 2.1) se observaron diferentes grupos varietales de arroz, de los cuales la variedad indica fue seleccionada para llevar a cabo este estudio una vez la misma era el grupo varietal con mayor número de individuos genotipados (451 individuos de un total de 738).

\begin{center}\includegraphics[width=1\linewidth]{TFM_files/figure-latex/unnamed-chunk-5-1} \end{center}

\begin{center}
\textbf{Figura 2.1:} Análisis de componentes principales en datos de arroz. Los puntos y las circuferencias de color representan distintos grupos varietales: tipo intermedio o mezclado (ADM), aromático (ARO), aus (AUS), indica (IND) y japónica (JAP).

\end{center}

En relación a los datos de fenotipo, el conjunto de datos proporciono información sobre distintos caracteres fenotípicos de relevancia agronómica como son la trillabilidad de la panícula, el peso del grano, la fuerza del culmo, entre otros (Figura 2.2), siendo seleccionada para este estudio el carácter tiempo de floración ya que en este se obervo suficiente variación fenotípica.

\begin{center}\includegraphics[width=1\linewidth]{TFM_files/figure-latex/unnamed-chunk-6-1} \end{center}

\begin{center}
\textbf{Figura 2.2:} Distribución de cada uno de los caracteres del conjunto de datos fenotípicos de arroz.

\end{center}

En lo que respecta a la información de pedigrí, esta no estaba disponible. Por ello, se utilizó la metodología implementada en el software MOLCOANC (Fernández y Toro \protect\hyperlink{ref-cite:24}{2006}) con el fin de contar con esta información. Este software . (Figura 2.3).

\begin{center}\includegraphics[width=1\linewidth]{TFM_files/figure-latex/unnamed-chunk-7-1} \end{center}

\begin{center}
\textbf{Figura 2.3:} .

\end{center}

\hypertarget{modelo-para-la-predicciuxf3n-genuxf3mica-y-habilidad-predictiva}{%
\subsection{Modelo para la predicción genómica y habilidad predictiva}\label{modelo-para-la-predicciuxf3n-genuxf3mica-y-habilidad-predictiva}}

Para llevar a cabo la predicción genómica mediante el mejor predictor lineal insesgado genómico de un solo paso (ssGBLUP), se eliminaron los loci de SNPs con una frecuencia del alelo menor de menos de 0.05. La predicción genómica se realizó mediante el siguiente modelo con los datos descritos anteriormente:

\begin{equation}
y = Za + e,
\end{equation}

donde \(y\) representa el valor del fenotipo a predecir (tiempo de floración) y \(Z\) es la matriz de incidencia que relaciona \(a\) con \(y\). El vector \(a\) representa los valores genotípicos como se describen en el siguiente parrafo, y \(e\) es el vector de residuos con una distribución que se asume normal con media igual a \(0\) y matriz de covarianza \(I\sigma^{2}_{e}\).

En la ecuación (1), \(a\)

Para identificar el efecto sobre la predictibilidad del tamaño de la muestra de entrenamiento, el número de datos de genotipado de SNPs y el número de individuos genotipados, se usaron diferentes subconjuntos de datos (Figura 2.4) con la siguientes características:

\begin{enumerate}
\def\labelenumi{\arabic{enumi}.}
\item
  Diferente información de pedigrí:
\item
  Diferentes densidades de SNPs:
\item
  Distinta cantidad de individuos genotipados:
\end{enumerate}

\begin{center}\includegraphics[width=1\linewidth]{TFM_files/figure-latex/unnamed-chunk-8-1} \end{center}

\begin{center}
\textbf{Figura 2.4:} Esquema del calculo de la matriz H a partir de las matrices A y G, con base en diferentes subconjuntos de datos. El recuadro 1 representa los tres pedigríes con diferentes número de individuos y que posteriormente se usaron para el calculo de la matriz A. El recuadro 2 representa matrices G con distinta dimensión dado el número de individuos genotipados. El recuadro 3 representa diferentes densidades de SNPs.

\end{center}

Se uso el coeficiente de correlación entre los valores fenotípicos observados y predichos como medida de la predictibilidad. De acuerdo a Xua, Zhub, y Zhang (\protect\hyperlink{ref-cite:25}{2014}), la predictibilidad debe obtenerse usando una muestra de validación independiente o mediante validación cruzada donde los individuos predichos no deben contribuir a la estimación de parámetros. En este sentido, el valor fenotípico observado de 48 del total de 451 individuos de la variedad indica (que corresponde a los individuos clasificados como variedades mejoradas) se considero como faltante.

\hypertarget{estudio-de-simulaciuxf3n}{%
\subsection{Estudio de simulación}\label{estudio-de-simulaciuxf3n}}

\begin{center}\includegraphics[width=1\linewidth]{TFM_files/figure-latex/unnamed-chunk-9-1} \end{center}

\hypertarget{results2}{%
\section{Resultados}\label{results2}}

\hypertarget{fenotipo-y-heredabilidad}{%
\subsection{Fenotipo y heredabilidad}\label{fenotipo-y-heredabilidad}}

\begin{center}
\textbf{Tabla 2.1:} Estimaciones de heredabilidad para el caracter tiempo de floración estimado por BLUP basado en el pedigrí.

\end{center}

\captionsetup[table]{labelformat=empty,skip=1pt}
\begin{longtable}{lrrr}
\toprule
 & \multicolumn{3}{c}{reml} \\ 
 \cmidrule(lr){2-4}
Parámetros & Pedigrí 1 & Pedigrí 2 & Pedigrí 3 \\ 
\midrule
Varianza aditiva & 0.49 & 0.46 & 0.57 \\ 
Varianza ambiental & 0.11 & 0.17 & 0.13 \\ 
Heredabilidad & 0.82 & 0.73 & 0.81 \\ 
 \bottomrule
\end{longtable}

\begin{center}\includegraphics[width=1\linewidth]{TFM_files/figure-latex/unnamed-chunk-11-1} \end{center}

\begin{center}
\textbf{Figura 2.5:} .

\end{center}

Los resultados del análisis de máxima verosimilitud restringida (REML) y\ldots{} (RKHS) bajo el modelo aditivo se observan en la Figura 2.5.

\begin{center}\includegraphics[width=1\linewidth]{TFM_files/figure-latex/unnamed-chunk-12-1} \end{center}

\begin{center}\includegraphics[width=1\linewidth]{TFM_files/figure-latex/unnamed-chunk-12-2} \end{center}

\begin{center}\includegraphics[width=1\linewidth]{TFM_files/figure-latex/unnamed-chunk-12-3} \end{center}

\begin{center}
\textbf{Figura 2.6:} .

\end{center}

\hypertarget{discussion2}{%
\section{Discusión}\label{discussion2}}

\hypertarget{appendix-appendix}{%
\appendix}


\hypertarget{anexo-del-capitulo-2}{%
\chapter{Anexo del capitulo 2}\label{anexo-del-capitulo-2}}

\begin{Shaded}
\begin{Highlighting}[]
\NormalTok{fn.mH <-}\StringTok{ }\ControlFlowTok{function}\NormalTok{(ped, mG) \{ }\CommentTok{# Esta función recibe como argu-}
                             \CommentTok{# mentos los datos con estructura }
                             \CommentTok{# (id | sire | dam | Gen (TRUE/FALSE)) }
                             \CommentTok{# y la matriz de relaciones genómicas.}
  
  \CommentTok{# 1. Se calcula la matriz de relaciones aditivas con base en }
  \CommentTok{# el pedigrí (A)}
  
\NormalTok{  ped_edit <-}\StringTok{ }\KeywordTok{editPed}\NormalTok{( }\CommentTok{# Esta función ordena el pedigrí.}
    \DataTypeTok{sire =}\NormalTok{ ped}\OperatorTok{$}\NormalTok{sire,}
    \DataTypeTok{dam =}\NormalTok{ ped}\OperatorTok{$}\NormalTok{dam,}
    \DataTypeTok{label =}\NormalTok{ ped}\OperatorTok{$}\NormalTok{id}
\NormalTok{    )}
\NormalTok{  pedi <-}\StringTok{ }\KeywordTok{pedigree}\NormalTok{( }\CommentTok{# Aquí se usa la salida anterior (ya orde-}
                    \CommentTok{# nado) y se crea un objeto de clase pedigree.}
    \DataTypeTok{sire =}\NormalTok{ ped_edit}\OperatorTok{$}\NormalTok{sire,}
    \DataTypeTok{dam =}\NormalTok{ ped_edit}\OperatorTok{$}\NormalTok{dam,}
    \DataTypeTok{label =}\NormalTok{ ped_edit}\OperatorTok{$}\NormalTok{label}
\NormalTok{    )}
\NormalTok{  Matrix_A <-}\StringTok{ }\KeywordTok{getA}\NormalTok{(}\DataTypeTok{ped =}\NormalTok{ pedi) }\CommentTok{# Esto dara la matriz de relaciones}
                               \CommentTok{# aditivas A.}
 
  \CommentTok{# 2. De lo anterior (Matriz_A) se extraen las partes correspon-}
  \CommentTok{# dientes a individuos no genotipados (1) y genotipados (2)}
  
  \CommentTok{# Individuos no genotipados:}
\NormalTok{  A_}\DecValTok{11}\NormalTok{ <-}\StringTok{ }\NormalTok{Matrix_A[ped}\OperatorTok{$}\NormalTok{Genotiped }\OperatorTok{!=}\StringTok{ }\DecValTok{1}\NormalTok{, ped}\OperatorTok{$}\NormalTok{Genotiped }\OperatorTok{!=}\StringTok{ }\DecValTok{1}\NormalTok{]}
  \CommentTok{# Individuos genotipados:}
\NormalTok{  A_}\DecValTok{22}\NormalTok{ <-}\StringTok{ }\NormalTok{Matrix_A[ped}\OperatorTok{$}\NormalTok{Genotiped }\OperatorTok{==}\StringTok{ }\DecValTok{1}\NormalTok{, ped}\OperatorTok{$}\NormalTok{Genotiped }\OperatorTok{==}\StringTok{ }\DecValTok{1}\NormalTok{]}
  \CommentTok{# Individuos no genotipados (en filas) y genotipados (en }
  \CommentTok{# columnas):}
\NormalTok{  A_}\DecValTok{12}\NormalTok{ <-}\StringTok{ }\NormalTok{Matrix_A[ped}\OperatorTok{$}\NormalTok{Genotiped }\OperatorTok{!=}\StringTok{ }\DecValTok{1}\NormalTok{, ped}\OperatorTok{$}\NormalTok{Genotiped }\OperatorTok{==}\StringTok{ }\DecValTok{1}\NormalTok{]}
  \CommentTok{# Transpuesta de la anterior (individuos no genotipados en }
  \CommentTok{# columnas y genotipados en filas):}
\NormalTok{  A_}\DecValTok{21}\NormalTok{ <-}\StringTok{ }\KeywordTok{t}\NormalTok{(A_}\DecValTok{12}\NormalTok{)}
  
  \CommentTok{# 3. Se coloca el nombre de las filas y y de las columnas }
  \CommentTok{# de la matriz G según los individuos genotipados}
  
  \KeywordTok{rownames}\NormalTok{(mG) <-}\StringTok{ }\NormalTok{ped}\OperatorTok{$}\NormalTok{id[ped}\OperatorTok{$}\NormalTok{Genotiped }\OperatorTok{==}\StringTok{ }\DecValTok{1}\NormalTok{]}
  \KeywordTok{colnames}\NormalTok{(mG) <-}\StringTok{ }\NormalTok{ped}\OperatorTok{$}\NormalTok{id[ped}\OperatorTok{$}\NormalTok{Genotiped }\OperatorTok{==}\StringTok{ }\DecValTok{1}\NormalTok{]}
  
  \CommentTok{# 4. Teniendo todos los componentes de la matriz H, se pro-}
  \CommentTok{# cede a su construcción y a calcular su inversa}
  
\NormalTok{  H_}\DecValTok{11}\NormalTok{ <-}\StringTok{ }\NormalTok{A_}\DecValTok{11} \OperatorTok{-}\StringTok{ }
\StringTok{    }\NormalTok{(A_}\DecValTok{12} \OperatorTok\StringTok{ }\KeywordTok{solve}\NormalTok{(A_}\DecValTok{22}\NormalTok{) }\OperatorTok\StringTok{ }\NormalTok{A_}\DecValTok{21}\NormalTok{) }\OperatorTok{+}\StringTok{ }
\StringTok{    }\NormalTok{(A_}\DecValTok{12} \OperatorTok\StringTok{ }\KeywordTok{solve}\NormalTok{(A_}\DecValTok{22}\NormalTok{) }\OperatorTok\StringTok{ }\NormalTok{mG }\OperatorTok\StringTok{ }\KeywordTok{solve}\NormalTok{(A_}\DecValTok{22}\NormalTok{) }\OperatorTok\StringTok{ }\NormalTok{A_}\DecValTok{21}\NormalTok{)}
\NormalTok{  H_}\DecValTok{12}\NormalTok{ <-}\StringTok{ }\NormalTok{A_}\DecValTok{12} \OperatorTok\StringTok{ }\KeywordTok{solve}\NormalTok{(A_}\DecValTok{22}\NormalTok{) }\OperatorTok\StringTok{ }\NormalTok{mG}
\NormalTok{  H_}\DecValTok{21}\NormalTok{ <-}\StringTok{ }\KeywordTok{t}\NormalTok{(H_}\DecValTok{12}\NormalTok{)}
\NormalTok{  H_}\DecValTok{22}\NormalTok{ <-}\StringTok{ }\NormalTok{mG}
  
\NormalTok{  H_}\DecValTok{11}\NormalTok{_H_}\DecValTok{12}\NormalTok{ <-}\StringTok{ }\KeywordTok{cbind}\NormalTok{(H_}\DecValTok{11}\NormalTok{, H_}\DecValTok{12}\NormalTok{)}
\NormalTok{  H_}\DecValTok{21}\NormalTok{_H_}\DecValTok{22}\NormalTok{ <-}\StringTok{ }\KeywordTok{cbind}\NormalTok{(H_}\DecValTok{21}\NormalTok{, H_}\DecValTok{22}\NormalTok{)}
\NormalTok{  mH <-}\StringTok{ }\KeywordTok{rbind}\NormalTok{(H_}\DecValTok{11}\NormalTok{_H_}\DecValTok{12}\NormalTok{, H_}\DecValTok{21}\NormalTok{_H_}\DecValTok{22}\NormalTok{)}
  
\NormalTok{  mH <-}\StringTok{ }\NormalTok{mH[}\KeywordTok{order}\NormalTok{(}\KeywordTok{as.numeric}\NormalTok{(}\KeywordTok{rownames}\NormalTok{(mH))), }
           \KeywordTok{order}\NormalTok{(}\KeywordTok{as.numeric}\NormalTok{(}\KeywordTok{colnames}\NormalTok{(mH)))]}
\NormalTok{  mH <-}\StringTok{ }\KeywordTok{Matrix}\NormalTok{(mH)}
  
  \CommentTok{# 5. Finalmente se indica retornar la inversa de la ma-}
  \CommentTok{# triz H (mH_1)}
  
  \KeywordTok{return}\NormalTok{(mH)}
\NormalTok{  \}}
\end{Highlighting}
\end{Shaded}

\begin{center}\includegraphics[width=1\linewidth]{TFM_files/figure-latex/unnamed-chunk-15-1} \end{center}

\begin{center}\includegraphics[width=1\linewidth]{TFM_files/figure-latex/unnamed-chunk-15-2} \end{center}

\begin{center}\includegraphics[width=1\linewidth]{TFM_files/figure-latex/unnamed-chunk-15-3} \end{center}

\begin{center}
\textbf{Figura A.1:}

\end{center}

\begin{center}\includegraphics[width=1\linewidth]{TFM_files/figure-latex/unnamed-chunk-16-1} \end{center}

\begin{center}\includegraphics[width=1\linewidth]{TFM_files/figure-latex/unnamed-chunk-16-2} \end{center}

\begin{center}\includegraphics[width=1\linewidth]{TFM_files/figure-latex/unnamed-chunk-16-3} \end{center}

\begin{center}
\textbf{Figura A.2:}

\end{center}

\begin{center}\includegraphics[width=1\linewidth]{TFM_files/figure-latex/unnamed-chunk-17-1} \end{center}

\begin{center}\includegraphics[width=1\linewidth]{TFM_files/figure-latex/unnamed-chunk-17-2} \end{center}

\begin{center}\includegraphics[width=1\linewidth]{TFM_files/figure-latex/unnamed-chunk-17-3} \end{center}

\begin{center}
\textbf{Figura A.3:}

\end{center}

\begin{center}\includegraphics[width=1\linewidth]{TFM_files/figure-latex/unnamed-chunk-18-1} \end{center}

\begin{center}\includegraphics[width=1\linewidth]{TFM_files/figure-latex/unnamed-chunk-18-2} \end{center}

\begin{center}\includegraphics[width=1\linewidth]{TFM_files/figure-latex/unnamed-chunk-18-3} \end{center}

\begin{center}
\textbf{Figura A.4:}

\end{center}

\begin{center}\includegraphics[width=1\linewidth]{TFM_files/figure-latex/unnamed-chunk-19-1} \end{center}

\begin{center}\includegraphics[width=1\linewidth]{TFM_files/figure-latex/unnamed-chunk-19-2} \end{center}

\begin{center}\includegraphics[width=1\linewidth]{TFM_files/figure-latex/unnamed-chunk-19-3} \end{center}

\begin{center}
\textbf{Figura A.5:}

\end{center}

\begin{center}\includegraphics[width=1\linewidth]{TFM_files/figure-latex/unnamed-chunk-20-1} \end{center}

\begin{center}\includegraphics[width=1\linewidth]{TFM_files/figure-latex/unnamed-chunk-20-2} \end{center}

\begin{center}\includegraphics[width=1\linewidth]{TFM_files/figure-latex/unnamed-chunk-20-3} \end{center}

\begin{center}
\textbf{Figura A.6:}

\end{center}

\backmatter

\hypertarget{bibliografuxeda}{%
\chapter*{Bibliografía}\label{bibliografuxeda}}
\addcontentsline{toc}{chapter}{Bibliografía}

\markboth{\MakeUppercase{Bibliography}}{} % have to explicitly state what to put in the heading (bug in bookdown?)
%format the references so they have a hanging indent. Remove these (and the \endgroup command) if you want regular indentation.
\begingroup
\hspace{\parindent}
\setlength{\parindent}{-0.25in}
\setlength{\leftskip}{0.25in}
\setlength{\parskip}{0pt}

\hypertarget{refs}{}
\leavevmode\hypertarget{ref-cite:21}{}%
Blasco, A. 2021. \emph{Mejora genética animal}. 1st edition. EDITORIAL SÍNTESIS, S. A.

\leavevmode\hypertarget{ref-cite:3}{}%
Blasco, A., y M. A. Toro. 2014. «A short critical history of the application of genomics to animal breeding». \emph{Livestock Science} 166: 4-9.

\leavevmode\hypertarget{ref-cite:10}{}%
Desta, Z. A., y R. Ortiz. 2014. «Genomic selection: genome-wide prediction in plant improvement». \emph{Trends in Plant Science} 19 (9): 592-601.

\leavevmode\hypertarget{ref-cite:24}{}%
Fernández, J., y M. Toro. 2006. «A new method to estimate relatedness from molecular markers». \emph{Molecular Ecology} 15: 1657-67.

\leavevmode\hypertarget{ref-cite:1}{}%
Fisher, R. A. 1918. «The correlaction between relatives under the supposition of Mendelian inheritance». \emph{Transactions of the Royal Society of Edinburgh} 52: 399-433.

\leavevmode\hypertarget{ref-cite:13}{}%
Hayes, B. J., P. M. Visscher, y M. E. Goddard. 2009. «Increased accuracy of artificial selection by using the realized relationship matrix». \emph{Genetics Research} 91: 47-60.

\leavevmode\hypertarget{ref-cite:20}{}%
Imai, A., T. Kuniga, T. Yoshioka, K. Nonaka, N. Mitani, H. Fukamachi, N. Hiehata, M. Yamamoto, y T. Hayashi. 2019. «Single-step genomic prediction of fruit-quality traits using phenotypic records of non-genotyped relatives in citrus». \emph{PLoS ONE} 14 (8). \url{https://doi.org/https://doi.org/10.1371/journal.pone.0221880}.

\leavevmode\hypertarget{ref-cite:11}{}%
Jurcic, E. J., P. V. Villalba, P. S. Pathauer, D. A. Palazzini, G. P. J. Oberschelp, L. Harrand, M. N. Garcia, et~al. 2021. «Genomic selection: genome-wide prediction in plant improvement». \emph{Trends in Plant Science} 127: 176-89.

\leavevmode\hypertarget{ref-cite:17}{}%
Legarra, A., I. Aguilar, y I. Misztal. 2009. «A relationship matrix including full pedigree and genomic information». \emph{Journal of Dairy Science} 92: 4656-63. \url{https://doi.org/10.3168/jds.2009-2061}.

\leavevmode\hypertarget{ref-cite:15}{}%
Legarra, A., O. F. Christensen, I. Aguilar, y I. Misztal. 2014. «Single Step, a general approach for genomic selection». \emph{Livestock Science}. \url{https://doi.org/http://dx.doi.org/10.1016/j.livsci.2014.04.029}.

\leavevmode\hypertarget{ref-cite:22}{}%
Lourenco, D., A. Legarra, S. Tsuruta, Y. Masuda, I. Aguilar, y I. Misztal. 2020. «Single-Step Genomic Evaluations from Theory to Practice: Using SNP Chips and Sequence Data in BLUPF90». \emph{Genes} 11: 790. \url{https://doi.org/doi:10.3390/genes11070790}.

\leavevmode\hypertarget{ref-cite:14}{}%
Misztal, I., S. E. Aggrrey, y W. M. Muir. 2012. «Experiences with a single-step genome evaluation». \emph{Poultry Science} 92: 2530-4.

\leavevmode\hypertarget{ref-cite:16}{}%
Misztal, I., A. Legarra, y I. Aguilar. 2009. «Computing procedures for genetic evaluation including phenotypic, full pedigree, and genomic information». \emph{Journal of Dairy Science} 92: 4648-55. \url{https://doi.org/10.3168/jds.2009-2064}.

\leavevmode\hypertarget{ref-cite:18}{}%
Misztal, I., D. Lourenco, y A. Legarra. 2020. «Current status of genomic evaluation». \emph{Journal of Animal Science} 98 (4): 1-14. \url{https://doi.org/10.1093/jas/skaa101}.

\leavevmode\hypertarget{ref-cite:6}{}%
Nakaya, A., y S. N. Isobe. 2012. «Will genomic selection be a practical method for plant breeding?» \emph{Annals of Botany} 110: 1303-16.

\leavevmode\hypertarget{ref-cite:2}{}%
Nelson, R. M., M. E. Pettersson, y Ö. Carlborg. 2012. «A century after Fisher: time for a new paradigm in quantitative genetics». \emph{Trends in Genetics} 29 (9): 669-76.

\leavevmode\hypertarget{ref-cite:5}{}%
Pérez-Enciso, M. 2017. «Animal breeding learning from machine learning». \emph{Journal of Animal Breeding and Genetics} 134: 85-86.

\leavevmode\hypertarget{ref-cite:19}{}%
Pérez-Rodríguez, P., J. Crossa, J. Rutkoski, J. Poland, R. Singh, A. Legarra, E. Autrique, J. Burgueño G. de los Campos, y S. Dreisigacker. 2017. «Single-step genomic and pedigree genotype x environment interaction models for predicting wheat lines in international environments». \emph{Plant Genome} 10 (2). \url{https://doi.org/10.3835/plantgenome2016.09.0089}.

\leavevmode\hypertarget{ref-cite:7}{}%
Tong, H., y Z. Nikoloski. 2021. «Machine learning approaches for crop improvement: leveraging phenotypic and genotypic big data». \emph{Journal of Plant Physiology} 257: 153354. \url{https://doi.org/10.1016/j.jplph.2020.153354}.

\leavevmode\hypertarget{ref-cite:9}{}%
Turelli, M. 2017. «Prediction of Total Genetic Value Using Genome-Wide Dense Marker Maps». \emph{Theoretical Population Biology} 118: 46-49.

\leavevmode\hypertarget{ref-cite:4}{}%
Villemereuil, P. de, H. Schielzeth, S. Nakagawa, y M. Morrissey. 2016. «General methods for evolutionary quantitative genetic inference from generalized mixed models». \emph{Genetics} 204: 1281-94.

\leavevmode\hypertarget{ref-cite:26}{}%
Vourlaki, I., R. Castanera, S. Ramos-Onsins, J. Casacuberta, y M. Pérez-Enciso. s.~f. «Transposable element polymorphisms improve prediction of complex agronomic traits in rice». \emph{Frontiers in Plant Science}.

\leavevmode\hypertarget{ref-cite:12}{}%
Wright, S. 1922. «Coefficients of inbreeding and relationship». \emph{The American Naturalist} 56: 330-38.

\leavevmode\hypertarget{ref-cite:25}{}%
Xua, S., D. Zhub, y Q. Zhang. 2014. «Predicting hybrid performance in rice using genomic best linear unbiased prediction». \emph{Proceedings of the National Academy of Sciences of the United States of America} 111 (34): 12456-61. \url{https://doi.org/10.1073/pnas.1413750111}.

\endgroup

\hypertarget{agradecimientos}{%
\chapter*{Agradecimientos}\label{agradecimientos}}
\addcontentsline{toc}{chapter}{Agradecimientos}

\chaptermark{Acknowledgments}

\includegraphics{figures/uvalogo_regular_p_en.pdf}

\backmatter

\end{document}
