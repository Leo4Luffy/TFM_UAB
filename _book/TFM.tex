% This is the default LaTeX template (default-1.17.0.2.tex) from the RMarkdown package, from:
% https://github.com/rstudio/rmarkdown/blob/master/inst/rmd/latex/default-1.17.0.2.tex
%
% New additions to the template are marked with "LCR"

%\documentclass[11pt,spanish,]{book} %LCR
 % if not, force the oneside and a4paper options, which seem to be the only reasonable defaults
\documentclass[11pt,spanish,a4paper,oneside,]{book} %LCR

\usepackage{lmodern}
\usepackage{amssymb,amsmath}
\usepackage{ifxetex,ifluatex}
\usepackage{fixltx2e} % provides \textsubscript
\ifnum 0\ifxetex 1\fi\ifluatex 1\fi=0 % if pdftex
  \usepackage[T1]{fontenc}
  \usepackage[utf8]{inputenc}
\else % if luatex or xelatex
  \ifxetex
    \usepackage{mathspec}
  \else
    \usepackage{fontspec}
  \fi
  \defaultfontfeatures{Ligatures=TeX,Scale=MatchLowercase}
\fi
% use upquote if available, for straight quotes in verbatim environments
\IfFileExists{upquote.sty}{\usepackage{upquote}}{}
% use microtype if available
\IfFileExists{microtype.sty}{%
\usepackage{microtype}
\UseMicrotypeSet[protrusion]{basicmath} % disable protrusion for tt fonts
}{}

 %LCR
\usepackage{hyperref}
\PassOptionsToPackage{usenames,dvipsnames}{color} % color is loaded by hyperref
\hypersetup{unicode=true,
            pdftitle={Evaluación de la predicción genómica de un solo paso en plantas},
            pdfauthor={Jorge Leonardo López Martínez},
            colorlinks=true,
            linkcolor=cyan,
            citecolor=Blue,
            urlcolor=cyan,
            breaklinks=true}
\urlstyle{same}  % don't use monospace font for urls
\ifnum 0\ifxetex 1\fi\ifluatex 1\fi=0 % if pdftex
  \usepackage[shorthands=off,main=spanish]{babel}
\else
\usepackage{polyglossia}
  \setmainlanguage{spanish}
  % Tabla en lugar de cuadro
  \gappto\captionsspanish{\renewcommand{\tablename}{Tabla}  
          \renewcommand{\listtablename}{Índice de tablas}}
\else
  \usepackage[spanish,es-tabla]{babel}
\fi
\usepackage{color}
\usepackage{fancyvrb}
\newcommand{\VerbBar}{|}
\newcommand{\VERB}{\Verb[commandchars=\\\{\}]}
\DefineVerbatimEnvironment{Highlighting}{Verbatim}{commandchars=\\\{\}}
% Add ',fontsize=\small' for more characters per line
\usepackage{framed}
\definecolor{shadecolor}{RGB}{248,248,248}
\newenvironment{Shaded}{\begin{snugshade}}{\end{snugshade}}
\newcommand{\AlertTok}[1]{\textcolor[rgb]{0.94,0.16,0.16}{#1}}
\newcommand{\AnnotationTok}[1]{\textcolor[rgb]{0.56,0.35,0.01}{\textbf{\textit{#1}}}}
\newcommand{\AttributeTok}[1]{\textcolor[rgb]{0.77,0.63,0.00}{#1}}
\newcommand{\BaseNTok}[1]{\textcolor[rgb]{0.00,0.00,0.81}{#1}}
\newcommand{\BuiltInTok}[1]{#1}
\newcommand{\CharTok}[1]{\textcolor[rgb]{0.31,0.60,0.02}{#1}}
\newcommand{\CommentTok}[1]{\textcolor[rgb]{0.56,0.35,0.01}{\textit{#1}}}
\newcommand{\CommentVarTok}[1]{\textcolor[rgb]{0.56,0.35,0.01}{\textbf{\textit{#1}}}}
\newcommand{\ConstantTok}[1]{\textcolor[rgb]{0.00,0.00,0.00}{#1}}
\newcommand{\ControlFlowTok}[1]{\textcolor[rgb]{0.13,0.29,0.53}{\textbf{#1}}}
\newcommand{\DataTypeTok}[1]{\textcolor[rgb]{0.13,0.29,0.53}{#1}}
\newcommand{\DecValTok}[1]{\textcolor[rgb]{0.00,0.00,0.81}{#1}}
\newcommand{\DocumentationTok}[1]{\textcolor[rgb]{0.56,0.35,0.01}{\textbf{\textit{#1}}}}
\newcommand{\ErrorTok}[1]{\textcolor[rgb]{0.64,0.00,0.00}{\textbf{#1}}}
\newcommand{\ExtensionTok}[1]{#1}
\newcommand{\FloatTok}[1]{\textcolor[rgb]{0.00,0.00,0.81}{#1}}
\newcommand{\FunctionTok}[1]{\textcolor[rgb]{0.00,0.00,0.00}{#1}}
\newcommand{\ImportTok}[1]{#1}
\newcommand{\InformationTok}[1]{\textcolor[rgb]{0.56,0.35,0.01}{\textbf{\textit{#1}}}}
\newcommand{\KeywordTok}[1]{\textcolor[rgb]{0.13,0.29,0.53}{\textbf{#1}}}
\newcommand{\NormalTok}[1]{#1}
\newcommand{\OperatorTok}[1]{\textcolor[rgb]{0.81,0.36,0.00}{\textbf{#1}}}
\newcommand{\OtherTok}[1]{\textcolor[rgb]{0.56,0.35,0.01}{#1}}
\newcommand{\PreprocessorTok}[1]{\textcolor[rgb]{0.56,0.35,0.01}{\textit{#1}}}
\newcommand{\RegionMarkerTok}[1]{#1}
\newcommand{\SpecialCharTok}[1]{\textcolor[rgb]{0.00,0.00,0.00}{#1}}
\newcommand{\SpecialStringTok}[1]{\textcolor[rgb]{0.31,0.60,0.02}{#1}}
\newcommand{\StringTok}[1]{\textcolor[rgb]{0.31,0.60,0.02}{#1}}
\newcommand{\VariableTok}[1]{\textcolor[rgb]{0.00,0.00,0.00}{#1}}
\newcommand{\VerbatimStringTok}[1]{\textcolor[rgb]{0.31,0.60,0.02}{#1}}
\newcommand{\WarningTok}[1]{\textcolor[rgb]{0.56,0.35,0.01}{\textbf{\textit{#1}}}}
\usepackage{longtable,booktabs}
\usepackage{graphicx,grffile}
\makeatletter
\def\maxwidth{\ifdim\Gin@nat@width>\linewidth\linewidth\else\Gin@nat@width\fi}
\def\maxheight{\ifdim\Gin@nat@height>\textheight\textheight\else\Gin@nat@height\fi}
\makeatother
% Scale images if necessary, so that they will not overflow the page
% margins by default, and it is still possible to overwrite the defaults
% using explicit options in \includegraphics[width, height, ...]{}
\setkeys{Gin}{width=\maxwidth,height=\maxheight,keepaspectratio}
% Make links footnotes instead of hotlinks:
\renewcommand{\href}[2]{#2\footnote{\url{#1}}}
\setlength{\emergencystretch}{3em}  % prevent overfull lines
\providecommand{\tightlist}{%
  \setlength{\itemsep}{0pt}\setlength{\parskip}{0pt}}
\setcounter{secnumdepth}{5}
% Redefines (sub)paragraphs to behave more like sections
\ifx\paragraph\undefined\else
\let\oldparagraph\paragraph
\renewcommand{\paragraph}[1]{\oldparagraph{#1}\mbox{}}
\fi
\ifx\subparagraph\undefined\else
\let\oldsubparagraph\subparagraph
\renewcommand{\subparagraph}[1]{\oldsubparagraph{#1}\mbox{}}
\fi

% LCR fix for new required cslreferences environment in pandoc
% from https://github.com/rstudio/rticles/pull/335/commits/a9937b6
% originally proposed by LS: https://github.com/LDSamson/amsterdown/commit/4d9841e
% Pandoc citation processing

%%% Use protect on footnotes to avoid problems with footnotes in titles
\let\rmarkdownfootnote\footnote%
\def\footnote{\protect\rmarkdownfootnote}

%%% This fixes a TexLive 2019 change that broke pandoc template. Will also be fixed in pandoc 2.8 %LCR
% https://github.com/jgm/pandoc/issues/5801
\renewcommand{\linethickness}{0.05em}

\usepackage{amsmath}
\usepackage{booktabs}
\usepackage{caption}
\usepackage{longtable}

%%%%%%%%%%%%% BEGIN DOCUMENT %%%%%%%%%%%%%
\begin{document}

%% Page I: the half-title / "Franse pagina" %LCR
\frontmatter
\thispagestyle{empty}
\def\drop{.1\textheight}

\vspace*{\drop}
\begin{center}
\Huge \textsc{Evaluación de la predicción genómica de un solo paso en plantas}
\end{center}

%% Page II: Colophon %LCR
\clearpage
\thispagestyle{empty}
\vspace*{\fill}
\begingroup % to change formatting only temporarily
\small
\setlength{\parskip}{\baselineskip} % add space between paragraphs
\setlength\parindent{0pt} % no indents

Esta tesis se escribio usando los paquetes de R (R) Markdown, \LaTeX\ , \verb+bookdown+  y \verb+amsterdown+.

\vspace{\baselineskip}
\includegraphics{_bookdown_files/CC-BY.png} \newline
Una versión en línea de esta tesis esta disponible en 
\url{https://github.com/Leo4Luffy/TFM_UAB},
bajo la licencia Creative Commons Attribution-NonCommercial-ShareAlike 4.0 International License.
\endgroup

%% Page III: `Title page' mandated by University of Amsterdam %LCR
\clearpage
\thispagestyle{empty}
\begin{center}
\includegraphics[width=41mm]{_bookdown_files/logo_uab.png} \includegraphics[width=41mm]{_bookdown_files/logo_upv.jpg} \includegraphics[width=41mm]{_bookdown_files/logo_ciheam.jpg} \newline
\vspace{\baselineskip}
\Huge{Evaluación de la predicción genómica de un solo paso en plantas}\par
\vspace{\baselineskip}
\linespread{1.3}{\normalsize Tesis académica para obtener\\
el grado de Máster en Mejora Genética y\\
Biotecnología de la Reproducción bajo la\\
dirección del prof. dr. Miguel Pérez Enciso\\ % make sure this is the current rector magnificus
\mbox{ante una comisión constituida por la Junta del Máster,}\\
para ser defendido en publico el\\
Colocar aquí la fecha de la defensa, a las colocar la hora aquí  \\ }\par %
\vspace{\baselineskip}
{\Large Jorge Leonardo López Martínez}\par
\vspace{\baselineskip}
\hfill\includegraphics[width=44mm]{_bookdown_files/logo_crag.png}\hspace*{\fill} \newline
\end{center}

%% Page IV: info on thesis committee %LCR
\clearpage
\thispagestyle{empty}
\noindent\textbf{Dirección:}\\
\\
\noindent\begin{tabular}{@{}lll}

Director:
&  prof. dr. M. Pérez-Enciso & Centre for Research in Agricultural Genomics\\

\\
\end{tabular}\\

%%%%%%%%%%%%%%%%%%


{
\hypersetup{linkcolor=black}
\setcounter{tocdepth}{1}
\tableofcontents
}
\mainmatter
\hypertarget{revisiuxf3n-de-literatura}{%
\chapter{Revisión de literatura}\label{revisiuxf3n-de-literatura}}

\hypertarget{breve-historia-hacia-la-selecciuxf3n-genuxf3mica}{%
\section{Breve historia hacia la selección genómica}\label{breve-historia-hacia-la-selecciuxf3n-genuxf3mica}}

La historia de la genética tanto cuantitativa como molecular se remonta a la contribución de muchas personas (Figura 1.1), hecho que permitió la conexión entre ambas disciplinas y el desarrollo de lo que hoy en día se conoce como selección genómica.

\begin{center}\includegraphics[width=1\linewidth]{figures/Crono} \end{center}

\begin{center}
\textbf{Figura 1.1:} Cronología de las disciplinas de la genética molecular y la genética cuantitativa. Varios descubrimientos permitieron la conexión entre ambas disciplinas lo que permitió el desarrollo de la selección genómica. Figura adaptada de Nelson, Pettersson, y Carlborg (\protect\hyperlink{ref-cite:2}{2012}).

\end{center}

La genética cuantitativa se formo hace más de un siglo en ausencia directa de datos genéticamente observables (Nelson, Pettersson, y Carlborg \protect\hyperlink{ref-cite:2}{2012}). Esta disciplina se formo gracias a los avances teóricos de Ronald Fisher quien proporcionó una teoría que hizo posible interpretar los descubrimientos de la genética biométrica dentro de los estudios de herencia Mendeliana, permitiendo con ello unificar las escuelas de pensamiento Mendeliano y biométrico que para ese entonces estaban en constante debate. Dicha teoría, denominada como teoría del modelo infinitesimal, supuso que la herencia genética es principalmente aditiva, y que la varianza genética de un carácter esta determinada por un gran número de factores Mendelianos (hoy en día conocidos como genes), cada uno de los cuales tiene una pequeña contribución al fenotipo del carácter (Nelson, Pettersson, y Carlborg \protect\hyperlink{ref-cite:2}{2012}; Turelli \protect\hyperlink{ref-cite:9}{2017}). A partir de este entonces, la genética cuantitativa fue extremadamente productiva a medida que fue adhiriéndose a la teoría del modelo infinitesimal.

Se denomina como valor de cría estimado (EBV) al efecto genético que un individuo posee y que puede transmitir a su descendencia. Este se puede predecir en función de un modelo que relaciona el fenotipo de una población con la información de pedigrí mediante el uso del mejor predictor lineal insesgado (BLUP) (Tong y Nikoloski \protect\hyperlink{ref-cite:7}{2021}). Este procedimiento fue resultado del esfuerzo de Charles Roy Henderson quien a inicio de la década de 1950 contribuyó a su desarrollo (Freeman \protect\hyperlink{ref-cite:28}{1991}; Searle \protect\hyperlink{ref-cite:29}{1991}; Schaeffer \protect\hyperlink{ref-cite:27}{1991}). A pesar que desde entonces el BLUP fue el método más utilizado para la mejora genética tanto en animales como en plantas, hoy en día se reconoce que dicho procedimiento ignora la base física de la herencia (el ADN), y utiliza una representación conceptual elemental de como la información genética es heredada (esto es, ambos progenitores deben aportar la mitad de la información genética a su descendencia) (Legarra et~al. \protect\hyperlink{ref-cite:15}{2014}).

En otro orden de ideas, el rápido desarrollo de la genética molecular a partir de los años 60 permitió comprender mejor los mecanismos de la herencia. Esta disciplina permitió, a diferencia de la genética cuantitativa, estudiar de forma directa el gen, lo que facilitó a finales de la década de 1970 e inicio de 1980 el descubrimiento de secuencias variables de ADN con fenotipos fácilmente observables (Legarra, Lourenco, y Vitezica \protect\hyperlink{ref-cite:30}{2018}). Son ejemplo de estas secuencias (denominadas como marcadores de ADN) los microsatélites, los polimorfismos en el tamaño de los fragmentos de restricción (RFLP) y los polimorfismos de un sólo nucleótido (SNP), siendo este último hoy en día el principal marcador utilizado para detectar variaciones en el ADN.

Dichos marcadores de ADN, al representar las diferencias en el ADN heredado por dos individuos (Legarra et~al. \protect\hyperlink{ref-cite:15}{2014}), abrieron la posibilidad de obtener una predicción más precisa de los EBV (Misztal, Aggrrey, y Muir \protect\hyperlink{ref-cite:14}{2012}; delosCampos et~al. \protect\hyperlink{ref-cite:31}{2013}), comparado al método BLUP mencionado en párrafos anteriores. Según los mismos autores (delosCampos et~al. \protect\hyperlink{ref-cite:31}{2013}), los primeros intentos de integrar datos de marcadores de ADN en las predicciones se basaron en el supuesto de que era posible encontrar genes que contribuyeran a la variación genética del carácter. Este enfoque, conocido como etiquetado de genes o mapeo de QTL, permitió identificar la genética subyacente a la variación fenotípica de un carácter (delosCampos et~al. \protect\hyperlink{ref-cite:31}{2013}; Legarra, Lourenco, y Vitezica \protect\hyperlink{ref-cite:30}{2018}; Qanbari \protect\hyperlink{ref-cite:36}{2020}).

Tanto en animales como en plantas, el interés principal en el mapeo de QTL consistió en usarse en un método conocido como selección asistida por marcadores (MAS) (Blasco y Toro \protect\hyperlink{ref-cite:3}{2014}), proceso en el cual los individuos portadores de un marcador de ADN deseado podían ser identificados y seleccionados para aumentar la respuesta genética de caracteres cuantitativos de relevancia económica (Kyselova, Tichý, y Jochová \protect\hyperlink{ref-cite:32}{2021}). Blasco y Toro (\protect\hyperlink{ref-cite:3}{2014}) describen la MAS como un proceso en el cual se detectan genes que afectan directamente un carácter (QTL), que al ser seleccionados, logran una mejora genética al aumentar su frecuencia (Figura 1.2).

\begin{center}\includegraphics[width=1\linewidth]{figures/MAS} \end{center}

\begin{center}
\textbf{Figura 1.2:} Esquema de la MAS. En la MAS, los fenotipos y genotipos de la población de mapeo se analizan usando un modelo estadístico, identificando con ello relaciones significativas entre fenotipos y genotipos. Por último, se seleccionan los individuos favorables con base en datos de genotipo. Figura adaptada de Nakaya y Isobe (\protect\hyperlink{ref-cite:6}{2012}).

\end{center}

Si bien la MAS abrió la posibilidad de investigar la variación genética en animales y en plantas, permitiendo también identificar genes que afectaban el desempeño de caracteres económicamente importantes, la literatura científica coincide en afirmar lo limitado que fue esta metodología al no detectar marcadores de ADN con efectos genéticos menores (Blasco y Toro \protect\hyperlink{ref-cite:3}{2014}; Desta y Ortiz \protect\hyperlink{ref-cite:10}{2014}; Kyselova, Tichý, y Jochová \protect\hyperlink{ref-cite:32}{2021}; Tong y Nikoloski \protect\hyperlink{ref-cite:7}{2021}). Y es que, como es sabido, la mayoría de los caracteres económicamente importantes son cuantitativos y complejos, lo que quiere decir que son caracteres controlados por muchos genes de pequeño efecto y/o por una combinación de genes mayores y menores, lo que hace del la MAS un método poco adecuado para este tipo de arquitectura genética de caracteres.

Finalmente, en el año 2001, Theodorus Meuwissen, Ben Hayes y Michael Goddard presentaron una alternativa a la MAS, superando con ello las limitaciones que suponía el uso de esta metodología. A esta nueva alternativa se le dio el nombre de selección genómica. Solo fue cuestión de tiempo para que los datos obtenidos de la genética molecular se integraran a los modelos estadísticos de la genética cuantitativa, permitiendo así el análisis de caracteres complejos en el marco de efectos del modelo infinitesimal.

\hypertarget{la-selecciuxf3n-genuxf3mica}{%
\section{La selección genómica}\label{la-selecciuxf3n-genuxf3mica}}

\hypertarget{definiciuxf3n-de-la-selecciuxf3n-genuxf3mica}{%
\subsection{Definición de la selección genómica}\label{definiciuxf3n-de-la-selecciuxf3n-genuxf3mica}}

Se denomina selección genómica a una serie de métodos que usan decenas de miles de marcadores de ADN, principalmente SNP, para realizar la predicción del EBV (aunque en selección genómica es común referirse al EBV como valor de cría basado en marcadores de ADN o GEBV). Blasco y Toro (\protect\hyperlink{ref-cite:3}{2014}) y Ahmadi et~al. (\protect\hyperlink{ref-cite:33}{2020}) describen este método como un proceso en el cual se usan grandes cantidades de marcadores de ADN para construir un modelo de relaciones genotipo-fenotipo en una población de entrenamiento. Luego el modelo de selección genómica resultante se utiliza en una población de prueba que solo está genotipada, y se predice en ella el GEBV con el que se lleva a cabo la selección (Figura 1.3). Por tanto, la selección genómica suele ser vista como una forma de MAS en la que se seleccionan individuos según el GEBV en lugar de pocos QTL (Nakaya y Isobe \protect\hyperlink{ref-cite:6}{2012}).

\begin{center}\includegraphics[width=1\linewidth]{figures/GS} \end{center}

\begin{center}
\textbf{Figura 1.3:} Esquema de la selección genómica. La selección genómica utiliza un modelo estadístico, diseñado a partir de datos genotípicos y fenotípicos en una población de entrenamiento, para predecir el GEBV de los individuos en una población de prueba con datos genotípicos. Por último, los individuos se seleccionan de acuerdo a su GEBV. Figura adaptada de Tong y Nikoloski (\protect\hyperlink{ref-cite:7}{2021}).

\end{center}

El uso de decenas de miles de marcadores de ADN es una de las características fundamentales de la selección genómica (Desta y Ortiz \protect\hyperlink{ref-cite:10}{2014}). Al contar con tal cantidad, la probabilidad de que algunos de estos marcadores estén en desequilibrio de ligamiento con el QTL tiende a aumentar (Meuwissen, Hayes, y Goddard \protect\hyperlink{ref-cite:8}{2001}), con lo cual, aún cuando dichos marcadores no tienen efecto biológico sobre el carácter, a partir de este hecho biológico si que se garantizaría una asociación (no observada) entre el QTL y el carácter (Legarra, Lourenco, y Vitezica \protect\hyperlink{ref-cite:30}{2018}; Grinberg, Orhobor, y King \protect\hyperlink{ref-cite:35}{2020}; Qanbari \protect\hyperlink{ref-cite:36}{2020}).

\hypertarget{muxe9todos-estaduxedsticos-en-la-selecciuxf3n-genuxf3mica}{%
\subsection{Métodos estadísticos en la selección genómica}\label{muxe9todos-estaduxedsticos-en-la-selecciuxf3n-genuxf3mica}}

En la selección genómica, la relación genotipo-fenotipo puede ser representada como un modelo lineal (Figura 1.4). Por tanto, el modelo de regresión lineal es un enfoque fundamental en la selección genómica (Nakaya y Isobe \protect\hyperlink{ref-cite:6}{2012}; delosCampos et~al. \protect\hyperlink{ref-cite:31}{2013}; Crossa et~al. \protect\hyperlink{ref-cite:37}{2017}).

\begin{center}\includegraphics[width=1\linewidth]{figures/Mod_RL} \end{center}

\begin{center}
\textbf{Figura 1.4:} Relación genotipo-fenotipo de individuos (circulos amarillo y azul) para un solo marcador. \(Y_{i}\) y \(x_{i1}\) denotan los fenotipos y genotipos, y \(\mu\) y \(g_{i}\) son los parámetros a determinar. Los genotipos bialélicos se codifican como 0 y 1, y los fenotipos se distribuyen de acuerdo a una normal. Figura adaptada de Nakaya y Isobe (\protect\hyperlink{ref-cite:6}{2012}).

\end{center}

Dicha relación genotipo-fenotipo se puede expresar de la forma:

\begin{equation}
y_{i} = \mu + \sum_{j = 1}^{p}x_{ij}g_{j} + e_{i},
\end{equation}

donde \(i\) (\(1, 2, 3, …, n\)) representa a los individuos, \(j\) (\(1, 2, 3, …, p\)) corresponde a los marcadores, \(y_{i}\) denota el fenotipo para el \emph{i}-ésimo individuo, \(\mu\) corresponde a la media de la población, \(x_{ij}\) representa al genotipo del \emph{j}-ésimo marcador en el \emph{i}-ésimo individuo, \(g_{j}\) corresponde al efecto del \emph{j}-ésimo marcador en el fenotipo, y \(e_{i}\) es el término del error.

Así mismo, el modelo anterior se puede expresar en notación matricial como:

\begin{equation}
y = Zg + e,
\end{equation}

donde \(y\) es un vector de longitud igual al número de individuos (\(1, 2, 3, …, n\)) que representa al fenotipo, \(Z\) es una matriz que indica si el marcador es homocigoto dominante, heterocigoto u homocigoto recesivo (por ejemplo, 2 si es homocigoto dominante, 1 si es heterocigoto y 0 si es homocigoto recesivo), \(g\) es un vector de efectos del marcador en el fenotipo (tratados aquí como efectos fijos), y \(e\) es el término del error. Luego el GEBV se puede predecir como \(\hat{g} = (Z'Z)^{- 1} Z'y\) mediante mínimos cuadrados.

Sin embargo, con el uso decenas de miles de marcadores de ADN para predecir el GEBV, al emplear el modelo lineal en selección genómica puede surgir un problema conocido como p grande y n pequeño (Nakaya y Isobe \protect\hyperlink{ref-cite:6}{2012}; delosCampos et~al. \protect\hyperlink{ref-cite:31}{2013}; Tong y Nikoloski \protect\hyperlink{ref-cite:7}{2021}), hecho que puede afectar el uso de la regresión por mínimos cuadrados, ya que esta solo se puede aplicar en situaciones en las que el número de observaciones es mayor al número de variables o predictores. En tal sentido, la selección genómica brinda la oportunidad de enfrentar el problema del p grande y n pequeño por medio del uso de modelos de regresión lineal alternativos (Figura 1.5).

\begin{center}\includegraphics[width=1\linewidth]{figures/Mod_GS} \end{center}

\begin{center}
\textbf{Figura 1.5:} Enfoques estadísticos de la selección genómica. Cada uno de estos enfoques suponen distintas distribuciones de efectos de los marcadores sobre el carácter. Figura adaptada de delosCampos et~al. (\protect\hyperlink{ref-cite:31}{2013}) y de Tong y Nikoloski (\protect\hyperlink{ref-cite:7}{2021}).

\end{center}

Al proponer la teoría de la selección genómica, Meuwissen, Hayes, y Goddard (\protect\hyperlink{ref-cite:8}{2001}) proporcionaron también una serie de métodos estadísticos como solución al problema planteado en el párrafo anterior, esto es, el mejor predictor lineal insesgado por regresión de crestas (rrBLUP), y los métodos Bayesianos BayesA y BayesB.

En relación al rrBLUP, este se puede expresar como un modelo lineal mixto:

\begin{equation}
y = Xb + Zg + e,
\end{equation}

donde \(y\), \(Z\) y \(e\) denotan los mismos términos del modelo (1.2), \(g\) es un vector de efectos del marcador en el fenotipo (tratados aquí como efectos aleatorios), \(X\) es una matriz que indica los efectos fijos y \(b\) es un vector de efector fijos. Luego la predicción del GEBV se realiza a partir de \(\hat{g} = (Z'Z + I \lambda)^{- 1} Z'y\), donde \(I\) es una matriz identidad y \(\lambda\) es un factor de penalización que se agrega a la diagonal de \(Z'Z\), y permite estimar cualquier número de efectos de marcador. Dicho factor de penalización se estima por máxima verosimilitud restringida (REML) como \(\frac{\sigma^{2}_{e}} {\sigma^{2}_{g}}\) , donde \(\sigma^{2}_{g}\) es la varianza del efecto del marcador y \(\sigma^{2}_{e}\) es la varianza del error residual. En el rrBLUP, se asume que todos los marcadores explican cantidades iguales de variación genética (esto es, varianza común para el efecto del marcador) y supone que sus efectos son normalmente distribuidos (Tong y Nikoloski \protect\hyperlink{ref-cite:7}{2021}).

Con respecto a los métodos Bayesianos, estos, a diferencia del método anterior, no asumen una distribución normal de los efectos de los marcadores, sino que en su lugar permiten que una parte de dichos marcadores tengan efectos importantes sobre el carácter, permitiendo así efectos del marcador diferentes (Medina et~al. \protect\hyperlink{ref-cite:38}{2021}).

Los modelos de regresión Bayesianos (BayesA y BayesB) propuestos por Meuwissen, Hayes, y Goddard (\protect\hyperlink{ref-cite:8}{2001}), se diferencian en los distintos supuestos sobre los efectos del marcador y sus distribuciones. De acuerdo a Blasco (\protect\hyperlink{ref-cite:21}{2021}), en BayesA se supone que los efectos de los marcadores se distribuyen de acuerdo a una distribución t de Student en lugar de una normal, permitiendo así que algunos marcadores tengan efectos grandes, otros medianos y otros pequeños. En cuanto a BayesB, este presenta los mismos supuestos de BayesA, sin embargo, a diferencia de este último, en BayesB se permite que una parte de los marcadores no tengan efecto alguno sobre el carácter (Blasco \protect\hyperlink{ref-cite:21}{2021}). De la misma manera, BayesA y BayesB se diferencian en el procedimiento de estimación: BayesA utiliza el método de cadenas de Markov Monte Carlo (MCMC) para \ldots{} (Tan et~al. \protect\hyperlink{ref-cite:34}{2017}).

Sobre la base de los dos modelos de regresión Bayesianos propuestos por Meuwissen, Hayes, y Goddard (\protect\hyperlink{ref-cite:8}{2001}), se han desarrollado una gran variedad de modelos Bayesianos para la predicción del GEBV. Por ejemplo en BayesC, se supone que todos los marcadores se distribuyen de forma normal (como la rrBBLUP), sin embargo, al igual que en BayesB, se permite que un porcentaje de los marcadores no tengan efecto sobre el carácter (Blasco \protect\hyperlink{ref-cite:21}{2021}). Por otro lado, en el LASSO Bayesiano se supone que el efecto del marcador obedece a una distribución de Laplace, distribución que comparte las mismas características de la t de Student en BayesA al suponer que todos los marcadores tienen un efecto distinto de cero y con diferentes varianzas (Tan et~al. \protect\hyperlink{ref-cite:34}{2017}).

Un modelo de regresión equivalente al rrGBLUP, denominado como mejor predictor lineal insesgado genómico (GBLUP), fue propuesto por VanRaden (\protect\hyperlink{ref-cite:39}{2007}). En el GBLUP, se utiliza una matriz de parentesco basada en marcadores de ADN (denominada como matriz G) en lugar de la matriz de parentesco basado en pedigríes (denominada como matriz A) del BLUP descrito por Henderson (\protect\hyperlink{ref-cite:41}{1975}). Luego la predicción del GEBV se realiza mediante un modelo lineal mixto, cuyas ecuaciones son:

\begin{equation}
\begin{bmatrix}
X'X & X'Z \\
Z'X & Z'Z + G^{-1} \lambda
\end{bmatrix}
\begin{bmatrix}
\hat{b} \\
\hat{g}
\end{bmatrix}
=
\begin{bmatrix}
X'y \\
Z'y
\end{bmatrix}
,
\end{equation}

donde \(X\), \(Z\), \(b\), \(g\) y \(\lambda\) denotan los mismos términos de los modelos (1.2) y (1.3), y \(G^{-1}\) corresponde a la inversa de la matriz G. El GBLUP, al igual que el rrBLUP, supone una varianza común para el efecto del marcador y que el efecto de los marcadores están normalmente distribuidos.

La idea de VanRaden (\protect\hyperlink{ref-cite:39}{2007}), al sustituir la matriz A por la matriz G, consistió en precisar el parentesco real entre individuos al momento de estimar los valores de cría. Según Blasco (\protect\hyperlink{ref-cite:21}{2021}), el parentesco que proviene al usar la matriz A es un parentesco esperado, lo cual puede no reflejar el porcentaje real de genes idénticos entre dos individuos emparentados, caso contrario al parentesco derivado al usar la matriz G que es observado, debido a lo cual si puede evidenciar de forma precisa el parentesco real entre dos individuos (Figura 1.6).

\begin{center}\includegraphics[width=1\linewidth]{figures/Ped} \end{center}

\begin{center}
\textbf{Figura 1.6:} Parentesco observado entre individuos emparentados.

\end{center}

En comparación a los métodos Bayesianos descritos anteriormente, en el GBLUP no es necesario usar una población de entrenamiento para estimar el efecto del marcador de ADN y luego predecir el GEBV. En su lugar, en el GBLUP se pueden colocar directamente a los individuos con fenotipo y sin fenotipo en el mismo modelo, y al mismo tiempo predecir el GEBV, y calcular su precisión (Tan et~al. \protect\hyperlink{ref-cite:34}{2017}). En cuanto a la velocidad de calculo, el GBLUP es mucho más rápido que los método Bayesianos, por lo que es más adecuado para obtener rápidamente el GEBV. Sin embargo, los métodos Bayesianos permiten incorporar al modelo información previa proveniente de múltiples estudios, siendo esto una ventaja de los mismos sobre otros métodos como el GBLUP. En tal sentido, delosCampos et~al. (\protect\hyperlink{ref-cite:31}{2013}) proporcionan algunos ejemplos sobre que tipo de información se podría incorporar a los efectos previos asignado a los marcadores, por mencionar algunos de ellos, la ubicación del marcador en el genoma, si dicha ubicación corresponde a una región codificante o no, y si el marcador esta en una región del genoma que alberga genes que pueden afectar una carácter de interés.

\hypertarget{de-la-selecciuxf3n-genuxf3mica-de-muxfaltiples-pasos-a-un-solo-paso}{%
\subsection{De la selección genómica de múltiples pasos a un solo paso}\label{de-la-selecciuxf3n-genuxf3mica-de-muxfaltiples-pasos-a-un-solo-paso}}

Para implementar los métodos de selección genómica mencionados anteriormente (rrBLUP, BayesA-B-C, LASSO Bayesiano y GBLUP), es necesario disponer de información genotípica y fenotípica ya que los modelos estadísticos generalmente se construyen en base a esta información. Esta situación puede ser desventajosa al momento de implementar la selección genómica, ya que por lo general no todos los individuos pueden genotiparse y en ocasiones (principalmente en animales) no se tienen valores fenotípicos para caracteres de interés (por ejemplo, la producción de leche en machos) (Legarra, Aguilar, y Misztal \protect\hyperlink{ref-cite:17}{2009}; delosCampos et~al. \protect\hyperlink{ref-cite:31}{2013}; Jurcic et~al. \protect\hyperlink{ref-cite:11}{2021}; Blasco \protect\hyperlink{ref-cite:21}{2021}). Como solución al problema de la falta de fenotipos, VanRaden (\protect\hyperlink{ref-cite:39}{2007}), con la implementación del GBLUP, propuso asignarle pseudo-fenotipos o valores de-regresados (estos son, valores fenotípicos estimados a partir de los EBV) a aquellos individuos con fenotipos faltantes, basándose en la información de sus parientes, permitiendo de esta forma implementar la selección genómica combinando los EBV y los genotipos a través de múltiples pasos (Figura 1.7) (Legarra, Aguilar, y Misztal \protect\hyperlink{ref-cite:17}{2009}; Misztal, Legarra, y Aguilar \protect\hyperlink{ref-cite:16}{2009}; Misztal, Aggrrey, y Muir \protect\hyperlink{ref-cite:14}{2012}; Legarra et~al. \protect\hyperlink{ref-cite:15}{2014}; Misztal, Lourenco, y Legarra \protect\hyperlink{ref-cite:18}{2020}).

\begin{center}\includegraphics[width=1\linewidth]{figures/BLUPs} \end{center}

\begin{center}
\textbf{Figura 1.7:} Esquema de comparación del BLUP, GBLUP y ssGBLUP. El GBLUP es un proceso de tres pasos en el que los individuos, con base en su información fenotípica y de pedigrí, son evaluados inicialmente mediante el BLUP; luego, a partir de los pseudo-fenotipos resultantes de esta evaluación inicial, se lleva a cabo un análisis genómico de los individuos genotipados mediante el GBLUP. En el ssGBLUP se simplifica este proceso al incorporar la información genómica (la matriz G) desde el primer paso.

\end{center}

Empero, esta forma de implementar la selección genómica en múltiples pasos es tendente a cometer errores (Misztal, Aggrrey, y Muir \protect\hyperlink{ref-cite:14}{2012}), además de presentar inconvenientes como son la perdida de información y la dificultad de generalizarse a caracteres múltiples y maternos (Legarra, Aguilar, y Misztal \protect\hyperlink{ref-cite:17}{2009}; Legarra et~al. \protect\hyperlink{ref-cite:15}{2014}). Conscientes de esto, Legarra, Aguilar, y Misztal (\protect\hyperlink{ref-cite:17}{2009}) simplificaron el proceso de varios pasos al desarrollar un método de selección genómica, en el que los fenotipos de los individuos genotipados y no genotipados se analizan conjuntamente para predecir su EBV o GEBV (Imai et~al. \protect\hyperlink{ref-cite:20}{2019}; Jurcic et~al. \protect\hyperlink{ref-cite:11}{2021}), método el cual se denominó como mejor predictor lineal insesgado genómico de un solo paso (ssGBLUP).

En el ssGBLUP se dispone de una matriz de parentesco genómica de individuos genotipados y no genotipados, denominada como matriz de parentesco combinada o matriz H (Figura 1.7). Esta matriz se obtiene combinando información de parentesco basado en marcadores de ADN (matriz G) entre individuos genotipados, e información de parentesco basado en pedigríes (matiz A) entre individuos genotipados y no genotipados (Imai et~al. \protect\hyperlink{ref-cite:20}{2019}). Con ello, el proceso anterior de múltiples pasos tiende a simplificarse al incorporar la información genómica desde el primer paso (Legarra et~al. \protect\hyperlink{ref-cite:15}{2014}; Misztal, Legarra, y Aguilar \protect\hyperlink{ref-cite:16}{2009}), sin la necesidad del calculo posterior de pseudo-fenotipos (Misztal, Lourenco, y Legarra \protect\hyperlink{ref-cite:18}{2020}).

El proceso de construcción de la matriz H es simple. De acuerdo a delosCampos et~al. (\protect\hyperlink{ref-cite:31}{2013}), el parentesco genómico de los individuos no genotipados se estima a partir de los que sí lo estan, usando un procedimiento de regresión lineal que predice los genotipos no observados como combinaciones lineales de los genotipos observados con coeficientes de regresión derivados de las relaciones basadas en el pedigrí. Dicho proceso se observa a continuación:

Al ser una forma de BLUP o GBLUP en el que la matriz A y G, respectivamente, es sustituida por la matriz H (Legarra, Aguilar, y Misztal \protect\hyperlink{ref-cite:17}{2009}; Legarra et~al. \protect\hyperlink{ref-cite:15}{2014}; Blasco \protect\hyperlink{ref-cite:21}{2021}), el ssGBLUP se puede adecuar con facilidad a caracteres múltiples y maternos (Blasco \protect\hyperlink{ref-cite:21}{2021}), además se adapta también a las herramientas informáticas ya desarrolladas en base al BLUP y GBLUP (Lourenco et~al. \protect\hyperlink{ref-cite:22}{2020}). Este hecho hace del ssGBLUP un método de uso rutinario para la selección genómica, donde ha demostrado que produce una predicción más precisa en comparación a los métodos BLUP y GBLUP ya mencionados (Misztal, Aggrrey, y Muir \protect\hyperlink{ref-cite:14}{2012}; Pérez-Rodríguez et~al. \protect\hyperlink{ref-cite:19}{2017}; Misztal, Lourenco, y Legarra \protect\hyperlink{ref-cite:18}{2020}).

\hypertarget{factores-que-pueden-afectar-la-habilidad-predictiva-de-la-selecciuxf3n-genuxf3mica}{%
\subsection{Factores que pueden afectar la habilidad predictiva de la selección genómica}\label{factores-que-pueden-afectar-la-habilidad-predictiva-de-la-selecciuxf3n-genuxf3mica}}

\hypertarget{breve-descripciuxf3n-de-la-mejora-genuxe9tica-en-arroz}{%
\section{Breve descripción de la mejora genética en arroz}\label{breve-descripciuxf3n-de-la-mejora-genuxe9tica-en-arroz}}

\hypertarget{diferencias-de-la-mejora-genuxe9tica-en-animales-y-en-plantas}{%
\subsection{Diferencias de la mejora genética en animales y en plantas}\label{diferencias-de-la-mejora-genuxe9tica-en-animales-y-en-plantas}}

Tras la domesticación de animales y el cultivo de plantas, la especie humana consiguió producir animales y plantas mejoradas. Esto a partir de la selección artificial. La clave del éxito de los antiguos domesticadores y cultivadores consistió en cruzar individuos portadores de caracteres deseables, al comprender que los descendientes podrían heredar estas características, pese a que para esos tiempos se desconocían los mecanismos biológicos de la herencia (Holland \protect\hyperlink{ref-cite:43}{2014}). Hoy en día, se conoce como mejora genética al proceso de mejorar los caracteres fenotípicos deseables en animales y plantas mediante selección artificial (Tong y Nikoloski \protect\hyperlink{ref-cite:7}{2021}), siendo la misma reconocida como una intrincada integración de ciencia y practicidad (Caligari y Brown \protect\hyperlink{ref-cite:42}{2017}).

En términos generales, la mejora genética se puede organizar en tres procesos, siendo estos la producción de variación genética, la selección entre la variación y la multiplicación para uso comercial (Figura 1.8).

\begin{center}\includegraphics[width=1\linewidth]{figures/Mej_Anim_Plan} \end{center}

\begin{center}
\textbf{Figura 1.8:} Esquema de la mejora genética en animales y en plantas. Figura adaptada de Hickey et~al. (\protect\hyperlink{ref-cite:44}{2017}).

\end{center}

El primer requisito en la mejora genética, como se mencionó en el párrafo anterior, consiste en producir variación genética de los caracteres que se desean a mejorar. Tanto en animales como en plantas, la producción de variación genética se da a través del apareamiento de individuos cuya expresión de caracteres es deseable (Caligari y Brown \protect\hyperlink{ref-cite:42}{2017}). De esta forma, mediante el proceso natural de reproducción sexual, se puede obtener una descendencia que contiene genes de interés heredados de los dos progenitores.

Una vez producida la variación genética, es necesario seleccionar individuos con mejor expresión de caracteres. Tanto en animales como en plantas, la selección se lleva a cabo de forma recurrente (en los denominados núcleos de selección) con la finalidad de aumentar la frecuencia de genes favorables. Sin embargo, los métodos de selección históricamente usados han sido distintos. Por un lado, en plantas se ha utilizado principalmente la MAS con el fin de identificar e incorporar genes beneficiosos, favoreciendo así que genes con efectos moderados a grandes hayan sido explotados más ampliamente en plantas que en animales. Por otro lado, en animales la mayoría de caracteres económicamente importantes han sido cuantitativos y complejos, lo cual obligo a los mejoradores genéticos a utilizar enfoques biométricos para predecir el EBV mediante la combinación de información fenotípica y de pedigrí (propiamente el BLUP), y con ello tomar las decisiones de selección (Hickey et~al. \protect\hyperlink{ref-cite:44}{2017}).

Justo después de aumentar la frecuencia de genes deseables, es necesario multiplicar o difundir el mérito genético medio obtenido de la población mejorada, y así facilitar que en las granjas comerciales los agricultores produzcan los caracteres mejorados (Blasco \protect\hyperlink{ref-cite:21}{2021}). Según Hickey et~al. (\protect\hyperlink{ref-cite:44}{2017}), una diferencia importante entre la mejora genética de animales y plantas, es que en animales la mejora difundida a las granjas comerciales no se recicla a los núcleos de selección, mientras que en plantas, al darle importancia a la selección de productos mejorados en forma de variedades vegetales, dichas variedades pueden ser usadas como progenitores en un ciclo nuevo de cultivo.

Por otro lado, desde el momento en que la selección genómica fue propuesta por Meuwissen, Hayes, y Goddard (\protect\hyperlink{ref-cite:8}{2001}), se adapto rápidamente a la mejora genética en animales, principalmente al sector ganadero. Sin embargo, el uso de la selección genómica en plantas se ha quedado atrás (Wang, Crossa, y Gai \protect\hyperlink{ref-cite:46}{2020}) y son varias las razones de ello:

1- Los métodos de mejora genética en animales y en plantas han divergido a lo largo de los años, lo cual implica que se requiera de tiempo para que los avances y las contribuciones realizadas en un campo se trasladen al otro (Hickey et~al. \protect\hyperlink{ref-cite:44}{2017}).

2- El genoma de muchas especies de plantas es más compleja al genoma de los animales. Los animales al ser individuos diploides, aportan a la descendencia solo uno de sus dos alelos, por lo cual es más fácil predecir en ellos cuan efectiva sera la selección, al suponer que de los distintos componentes de la varianza genética solo se heredara la varianza debido a efectos aditivos (esto es, la heredabilidad en el sentido estricto). Caso contrario sucede en las plantas cuya respuesta a la selección puede implicar, en caso de que la especie sea poliploide o se haya propagado de forma vegetativa, otros tipos de interacción como la dominancia entre dos alelos (Holland \protect\hyperlink{ref-cite:43}{2014}).

3- Varios mejoradores genéticos de plantas argumentan que se pueden obtener algunos de los beneficios esperados de la selección genómica a través del uso de otros métodos (Hickey et~al. \protect\hyperlink{ref-cite:44}{2017}).

4- Es costoso invertir en infraestructura computacional y de registro tanto de datos genotípicos como fenotípicos requeridos para implementar la selección genómica. El tamaño de las poblaciones en la cría de animales son mucho más pequeños que la mayoría de poblaciones en la cría de plantas. Si bien los costos de genotipado por individuo son cada vez más bajos, el costo total de genotipado al considerar todas las plantas es aún hoy en día demasiado alto para la mayoría de programas de mejora genética en plantas (Wang, Crossa, y Gai \protect\hyperlink{ref-cite:46}{2020}).

Pese a lo anterior, en la literatura científica (Tabla1.1) se evidencia el uso potencial de la selección genómica para mejorar el mérito genético medio por selección tanto en animales como en plantas. A pesar de sus diferencias, ambas disciplinas requieren de conceptos y herramientas similares de selección genómica. Por lo tanto, es de esperar que los mejoradores genéticos de animales y de plantas se beneficien del trabajo conjunto para solucionar aquellos problemas que les son comunes.

\begin{center}
\textbf{Tabla 1.1:} .

\end{center}

\captionsetup[table]{labelformat=empty,skip=1pt}
\begin{longtable}{lllll}
\toprule
Especie & Carácter & Habilidad predictiva & Modelo & Referencia \\ 
\midrule
 &  &  &  &  \\ 
 \bottomrule
\end{longtable}

\hypertarget{relevancia-de-la-mejora-genuxe9tica-en-arroz}{%
\subsection{Relevancia de la mejora genética en arroz}\label{relevancia-de-la-mejora-genuxe9tica-en-arroz}}

\hypertarget{titulo}{%
\chapter{Titulo}\label{titulo}}

\textbf{Resumen}

\noindent 
Insert abstract.

\begin{center}\rule{0.5\linewidth}{0.5pt}\end{center}

\vspace*{\fill}

\noindent
\emph{Possibly insert citation here.}
\newpage

\hypertarget{intro2}{%
\section{Introducción}\label{intro2}}

La teoría de la genética en el estudio de caracteres cuantitativos se estableció hace más de un siglo cuando Ronald Fisher presentó un documento (Fisher \protect\hyperlink{ref-cite:1}{1918}) donde dio a conocer el desarrollo de la teoría del modelo infinitesimal, permitiendo con ello unificar dos de las escuelas de pensamiento que para ese entonces estaban en constante debate: la escuela de pensamiento Mendeliano, cuyo objetivo consistía en localizar y caracterizar factores de herencia, y la escuela de pensamiento biométrico, cuyo origen se remonta a Galton quien buscaba aplicar modelos biométricos con el fin de estudiar las relaciones entre parientes (Nelson, Pettersson, y Carlborg \protect\hyperlink{ref-cite:2}{2012}; Blasco y Toro \protect\hyperlink{ref-cite:3}{2014}).

La teoría del modelo infinitesimal desarrollado por Fisher establece que la varianza genética de un carácter esta determinado por un gran número de factores Mendelianos, cada uno de los cuales tiene una pequeña contribución aditiva al fenotipo de dicho carácter (Nelson, Pettersson, y Carlborg \protect\hyperlink{ref-cite:2}{2012}; Turelli \protect\hyperlink{ref-cite:9}{2017}). Naturalmente, los modelos usados en estudios de mejoramiento genético han sido concebidos en base a esta teoría (Villemereuil et~al. \protect\hyperlink{ref-cite:4}{2016}; Pérez-Enciso \protect\hyperlink{ref-cite:5}{2017}), siendo ejemplo de ello el mejor predictor lineal insesgado (BLUP) y el mejor predictor lineal insesgado genómico (GBLUP).

En las ciencias animales, el valor de cría estimado (EBV) se suele predecir en función de un conjunto de modelos que relacionan el fenotipo de una población con la información del pedigrí, mediante el uso del BLUP. No obstante, este método no es factible para poblaciones sin información de pedigrí o con una estructura poblacional compleja, como suele ser el caso de las plantas (Nakaya y Isobe \protect\hyperlink{ref-cite:6}{2012}; Tong y Nikoloski \protect\hyperlink{ref-cite:7}{2021}). Para el año 2001, Meuwissen, Hayes y Goddard propusieron un método innovador para predecir los valores de cría basado en marcadores de ADN (GEBV), denominándose tiempo después como selección genómica (Nakaya y Isobe \protect\hyperlink{ref-cite:6}{2012}; Blasco y Toro \protect\hyperlink{ref-cite:3}{2014}), el cual permitió también superar las limitaciones que suponía el uso del BLUP para predecir los valores de cría en plantas.

Hoy en día, la selección genómica se considera como un método potencial para el mejoramiento genético en plantas (Nakaya y Isobe \protect\hyperlink{ref-cite:6}{2012}), ya que sus ciclos reproductivos suelen ser prolongados, por lo cual con el uso de la selección genómica es posible acelerar dichos ciclos reproductivos con el beneficio adicional de mejorar la tasa de ganancia genética anual por unidad de tiempo y costo (Desta y Ortiz \protect\hyperlink{ref-cite:10}{2014}; Jurcic et~al. \protect\hyperlink{ref-cite:11}{2021}). Además, los datos sobre marcadores de ADN en todo el genoma están cada vez más disponibles para cultivos de relevancia agronómica (Tong y Nikoloski \protect\hyperlink{ref-cite:7}{2021}).

El GBLUP es uno de los métodos más comunes de selección genómica (Jurcic et~al. \protect\hyperlink{ref-cite:11}{2021}). De hecho, es el método más popular debido a su simplicidad al sustituir la matriz de relación de parentesco basado en pedigríes (Wright \protect\hyperlink{ref-cite:12}{1922}) por una matriz de relación basada en marcadores de ADN (Hayes, Visscher, y Goddard \protect\hyperlink{ref-cite:13}{2009}). Así mismo, el GBLUP predice con mayor precisión los GEBV en comparación a los EBV del BLUP, debido a que con el primero se estima mejor las relaciones entre individuos (Misztal, Aggrrey, y Muir \protect\hyperlink{ref-cite:14}{2012}), por lo cual la matriz de las relaciones genómicas suele verse como un estimador mejorado de las relaciones basadas en marcadores en lugar de pedigríes (Legarra et~al. \protect\hyperlink{ref-cite:15}{2014}).

En términos generales, la selección genómica es un proceso de tres pasos en el que los individuos, sobre la base de su información fenotípica y de pedigrí, son evaluados inicialmente mediante una evaluación genética tradicional por medio del BLUP, y posteriormente a partir de los fenotipos corregidos o pseudo-fenotipos resultantes de esta evaluación genética inicial, es llevado a cabo un análisis genómico de los individuos genotipados mediante el GBLUP. Por último y en base a la información generada, se calculan los GEBV por medio de un índice de selección (Legarra, Aguilar, y Misztal \protect\hyperlink{ref-cite:17}{2009}; Misztal, Legarra, y Aguilar \protect\hyperlink{ref-cite:16}{2009}; Misztal, Aggrrey, y Muir \protect\hyperlink{ref-cite:14}{2012}; Legarra et~al. \protect\hyperlink{ref-cite:15}{2014}; Misztal, Lourenco, y Legarra \protect\hyperlink{ref-cite:18}{2020}).

Como no todos los individuos pueden genotiparse, la selección genómica se lleva a cabo a partir del proceso anterior de tres pasos (Legarra, Aguilar, y Misztal \protect\hyperlink{ref-cite:17}{2009}). Sin embargo, este proceso es tendente a cometer errores (Misztal, Aggrrey, y Muir \protect\hyperlink{ref-cite:14}{2012}), además de presentar inconvenientes como son la perdida de información y la difícultad de generalizarse a caracteres múltiples y maternos (Legarra, Aguilar, y Misztal \protect\hyperlink{ref-cite:17}{2009}; Legarra et~al. \protect\hyperlink{ref-cite:15}{2014}). Conscientes de esto, Legarra, Aguilar, y Misztal (\protect\hyperlink{ref-cite:17}{2009}) simplificaron el proceso de varios pasos al desarrollar un método de selección genómica, en el que los fenotipos de los individuos genotipados y no genotipados se analizan conjuntamente para predecir sus valores de cría (Imai et~al. \protect\hyperlink{ref-cite:20}{2019}; Jurcic et~al. \protect\hyperlink{ref-cite:11}{2021}), método el cual se denomino como mejor predictor lineal insesgado genómico de un solo paso (ssGBLUP).

En el ssGBLUP se dispone de una matriz de parentesco genómica global de individuos genotipados y no genotipados, denominada como matriz de relación combinada o matriz H. Esta matriz se obtiene combinando información de la relación genómica entre individuos genotipados, e información de pedigrí entre individuos genotipados y no genotipados (Imai et~al. \protect\hyperlink{ref-cite:20}{2019}). Con ello, el proceso anterior de tres pasos tiende a simplificarse al incorporar la información genómica desde el primer paso (Legarra et~al. \protect\hyperlink{ref-cite:15}{2014}; Misztal, Legarra, y Aguilar \protect\hyperlink{ref-cite:16}{2009}), sin la necesidad del calculo posterior de fenotipos corregidos y la construcción del índice de selección mencionado previamente (Misztal, Lourenco, y Legarra \protect\hyperlink{ref-cite:18}{2020}).

Al ser una forma de BLUP en el que la matriz de relación de parentesco es sustituida por la matriz de relación combinada (Legarra, Aguilar, y Misztal \protect\hyperlink{ref-cite:17}{2009}; Legarra et~al. \protect\hyperlink{ref-cite:15}{2014}; Blasco \protect\hyperlink{ref-cite:21}{2021}), el ssGBLUP se puede adecuar con facilidad a caracteres múltiples y maternos (Blasco \protect\hyperlink{ref-cite:21}{2021}), además se adapta también a las herramientas informaticas ya desarrolladas en base al BLUP (Lourenco et~al. \protect\hyperlink{ref-cite:22}{2020}). Este hecho hace del ssGBLUP un método de uso rutinario para la evaluación genómica en animales, donde ha demostrado que produce una predicción más precisa de los valores de cría en comparación a los métodos BLUP y GBLUP antes mencionados (Misztal, Aggrrey, y Muir \protect\hyperlink{ref-cite:14}{2012}; Pérez-Rodríguez et~al. \protect\hyperlink{ref-cite:19}{2017}; Misztal, Lourenco, y Legarra \protect\hyperlink{ref-cite:18}{2020}). No obstante, el uso del ssGBLUP para la selección genómica en plantas es más reciente y escaso (Pérez-Rodríguez et~al. \protect\hyperlink{ref-cite:19}{2017}; Jurcic et~al. \protect\hyperlink{ref-cite:11}{2021}). En consecuencia, el \textbf{objetivo}

\hypertarget{methods2}{%
\section{Métodos}\label{methods2}}

\hypertarget{recurso-vegetal-y-datos-fenotuxedpicos}{%
\subsection{Recurso vegetal y datos fenotípicos}\label{recurso-vegetal-y-datos-fenotuxedpicos}}

Los conjuntos de datos se obtuvieron del \href{https://snp-seek.irri.org/index.zul;jsessionid=DD991975FDC4F320BE3C33ED056D0363}{Rice SNP-Seek Database}, el cual es un cibersitio con información sobre datos de genotipado de SNP y de fenotipos de distintas variedades de arroz (\emph{Oryza sativa L.}). Posteriormente, dichos conjuntos de datos fueron usados por Vourlaki et~al. (\protect\hyperlink{ref-cite:26}{s.~f.}), quienes sometieron los datos de genotipado de SNP a procedimientos de control de calidad, en los que fueron eliminados loci de SNP con una frecuencia del alelo menor de menos de 0.01 y con una tasa de ausencia mayor a 0.01.

Mediante un análisis de componentes principales realizado sobre los datos de genotipado de SNP (Figura 2.1) se observaron diferentes grupos varietales de arroz, de los cuales la variedad indica fue seleccionada para llevar a cabo este estudio una vez la misma era el grupo varietal con mayor número de individuos genotipados (451 individuos de un total de 738).

\begin{center}\includegraphics[width=1\linewidth]{figures/Graf_PCA} \end{center}

\begin{center}
\textbf{Figura 2.1:} Análisis de componentes principales en datos de arroz. Los puntos y las circuferencias de color representan distintos grupos varietales: tipo intermedio o mezclado (ADM), aromático (ARO), aus (AUS), indica (IND) y japónica (JAP).

\end{center}

En relación a los datos de fenotipo, el conjunto de datos proporciono información sobre distintos caracteres fenotípicos de relevancia agronómica como son la trillabilidad de la panícula, el peso del grano, la fuerza del culmo, entre otros (Figura 2.2), siendo seleccionada para este estudio el carácter tiempo de floración ya que en este se obervo suficiente variación fenotípica.

\begin{center}\includegraphics[width=1\linewidth]{figures/Graf_feno} \end{center}

\begin{center}
\textbf{Figura 2.2:} Distribución de cada uno de los caracteres del conjunto de datos fenotípicos de arroz.

\end{center}

En lo que respecta a la información de pedigrí, esta no estaba disponible. Por ello, se utilizó la metodología implementada en el software MOLCOANC (Fernández y Toro \protect\hyperlink{ref-cite:24}{2006}) con el fin de contar con esta información. Este software . (Figura 2.3).

\begin{center}\includegraphics[width=1\linewidth]{TFM_files/figure-latex/unnamed-chunk-18-1} \end{center}

\begin{center}
\textbf{Figura 2.3:} .

\end{center}

\hypertarget{modelo-para-la-predicciuxf3n-genuxf3mica-y-habilidad-predictiva}{%
\subsection{Modelo para la predicción genómica y habilidad predictiva}\label{modelo-para-la-predicciuxf3n-genuxf3mica-y-habilidad-predictiva}}

Para llevar a cabo la predicción genómica mediante el mejor predictor lineal insesgado genómico de un solo paso (ssGBLUP), se eliminaron los loci de SNP con una frecuencia del alelo menor de menos de 0.05. La predicción genómica se realizó mediante el siguiente modelo con los datos descritos anteriormente:

\begin{equation}
y = Za + e,
\end{equation}

donde \(y\) representa el valor del fenotipo a predecir (tiempo de floración) y \(Z\) es la matriz de incidencia que relaciona \(a\) con \(y\). El vector \(a\) representa los valores genotípicos como se describen en el siguiente parrafo, y \(e\) es el vector de residuos con una distribución que se asume normal con media igual a \(0\) y matriz de covarianza \(I\sigma^{2}_{e}\).

En la ecuación (1), \(a\)

Para identificar el efecto sobre la predictibilidad del tamaño de la muestra de entrenamiento, el número de datos de genotipado de SNP y el número de individuos genotipados, se usaron diferentes subconjuntos de datos (Figura 2.4) con la siguientes características:

\begin{enumerate}
\def\labelenumi{\arabic{enumi}.}
\item
  Diferente información de pedigrí:
\item
  Diferentes densidades de SNP:
\item
  Distinta cantidad de individuos genotipados:
\end{enumerate}

\begin{center}\includegraphics[width=1\linewidth]{TFM_files/figure-latex/unnamed-chunk-19-1} \end{center}

\begin{center}
\textbf{Figura 2.4:} Esquema del calculo de la matriz H a partir de las matrices A y G, con base en diferentes subconjuntos de datos. El recuadro 1 representa los tres pedigríes con diferentes número de individuos y que posteriormente se usaron para el calculo de la matriz A. El recuadro 2 representa diferentes densidades de SNP. El recuadro 3 representa matrices G con distinta dimensión dado el número de individuos genotipados.

\end{center}

Se uso el coeficiente de correlación entre los valores fenotípicos observados y predichos como medida de la predictibilidad. De acuerdo a Xua, Zhub, y Zhang (\protect\hyperlink{ref-cite:25}{2014}), la predictibilidad debe obtenerse usando una muestra de validación independiente o mediante validación cruzada donde los individuos predichos no deben contribuir a la estimación de parámetros. En este sentido, el valor fenotípico observado de 48 del total de 451 individuos de la variedad indica (que corresponde a los individuos clasificados como variedades mejoradas) se considero como faltante.

\hypertarget{estudio-de-simulaciuxf3n}{%
\subsection{Estudio de simulación}\label{estudio-de-simulaciuxf3n}}

\begin{center}\includegraphics[width=1\linewidth]{TFM_files/figure-latex/unnamed-chunk-20-1} \end{center}

\hypertarget{results2}{%
\section{Resultados}\label{results2}}

\hypertarget{fenotipo-y-heredabilidad}{%
\subsection{Fenotipo y heredabilidad}\label{fenotipo-y-heredabilidad}}

\begin{center}
\textbf{Tabla 2.1:} Estimaciones de heredabilidad para el caracter tiempo de floración estimado por BLUP basado en el pedigrí.

\end{center}

\captionsetup[table]{labelformat=empty,skip=1pt}
\begin{longtable}{lrrrccc}
\toprule
 & \multicolumn{3}{c}{Bayesiano} & \multicolumn{3}{c}{Penalizado} \\ 
 \cmidrule(lr){2-4} \cmidrule(lr){5-7}
Parámetros & Ped. 1\textsuperscript{1} & Ped. 2 & Ped. 3 & Ped. 1 & Ped. 2 & Ped. 3 \\ 
\midrule
Varianza aditiva & 0.43 & 0.50 & 0.56 & NA & NA & NA \\ 
Varianza ambiental & 0.16 & 0.14 & 0.11 & NA & NA & NA \\ 
Heredabilidad & 0.72 & 0.78 & 0.83 & NA & NA & NA \\ 
 \bottomrule
\end{longtable}
\vspace{-5mm}
\begin{minipage}{\linewidth}
\textsuperscript{1}Ped. 1 indica Pedigrí 1 \\ 
\end{minipage}

\begin{center}\includegraphics[width=1\linewidth]{figures/covar_h2} \end{center}

\begin{center}
\textbf{Figura 2.5:} .

\end{center}

Los resultados del análisis de máxima verosimilitud restringida (REML) y\ldots{} (RKHS) bajo el modelo aditivo se observan en la Figura 2.6.

\begin{center}\includegraphics[width=1\linewidth]{figures/Cor_Bay_Pen} \end{center}

\begin{center}
\textbf{Figura 2.6:} .

\end{center}

\begin{center}\includegraphics[width=1\linewidth]{figures/Cor_F2_F3} \end{center}

\begin{center}
\textbf{Figura 2.7:} .

\end{center}

\hypertarget{discussion2}{%
\section{Discusión}\label{discussion2}}

\hypertarget{appendix-appendix}{%
\appendix}


\hypertarget{anexos-del-capitulo-2}{%
\chapter{Anexos del capitulo 2}\label{anexos-del-capitulo-2}}

\hypertarget{funciuxf3n-para-el-calculo-de-la-matriz-de-relaciuxf3n-combinada}{%
\section{Función para el calculo de la matriz de relación combinada}\label{funciuxf3n-para-el-calculo-de-la-matriz-de-relaciuxf3n-combinada}}

\begin{Shaded}
\begin{Highlighting}[]
\NormalTok{fn.mH <-}\StringTok{ }\ControlFlowTok{function}\NormalTok{(ped, mG) \{ }\CommentTok{# Esta función recibe como argu-}
                             \CommentTok{# mentos los datos con estructu- }
                             \CommentTok{# ra (id | sire | dam | Gen (TRUE }
                             \CommentTok{# /FALSE)) y la matriz de relacio-}
                             \CommentTok{# nes genómicas.}
  
  \CommentTok{# 1. Se calcula la matriz de relaciones aditivas con base en }
  \CommentTok{# el pedigrí (A)}
  
\NormalTok{  ped_edit <-}\StringTok{ }\NormalTok{pedigreemm}\OperatorTok{::}\KeywordTok{editPed}\NormalTok{( }\CommentTok{# Esta función ordena el pe-}
                                   \CommentTok{# digrí.}
    \DataTypeTok{sire =}\NormalTok{ ped}\OperatorTok{$}\NormalTok{sire,}
    \DataTypeTok{dam =}\NormalTok{ ped}\OperatorTok{$}\NormalTok{dam,}
    \DataTypeTok{label =}\NormalTok{ ped}\OperatorTok{$}\NormalTok{id}
\NormalTok{    )}
\NormalTok{  pedi <-}\StringTok{ }\NormalTok{pedigreemm}\OperatorTok{::}\KeywordTok{pedigree}\NormalTok{( }\CommentTok{# Aquí se usa la salida anterior}
                                \CommentTok{# (ya ordenado) y se crea un ob-}
                                \CommentTok{# jeto de clase pedigree.}
    \DataTypeTok{sire =}\NormalTok{ ped_edit}\OperatorTok{$}\NormalTok{sire,}
    \DataTypeTok{dam =}\NormalTok{ ped_edit}\OperatorTok{$}\NormalTok{dam,}
    \DataTypeTok{label =}\NormalTok{ ped_edit}\OperatorTok{$}\NormalTok{label}
\NormalTok{    )}
\NormalTok{  Matrix_A <-}\StringTok{ }\NormalTok{pedigreemm}\OperatorTok{::}\KeywordTok{getA}\NormalTok{(}\DataTypeTok{ped =}\NormalTok{ pedi) }\CommentTok{# Esto dara la matriz}
                                           \CommentTok{# de relaciones adi-}
                                           \CommentTok{# tivas A.}
 
  \CommentTok{# 2. De lo anterior (Matriz_A) se extraen las partes correspon-}
  \CommentTok{# dientes a individuos no genotipados (1) y genotipados (2)}
  
  \CommentTok{# Individuos no genotipados:}
\NormalTok{  A_}\DecValTok{11}\NormalTok{ <-}\StringTok{ }\NormalTok{Matrix_A[ped}\OperatorTok{$}\NormalTok{Genotiped }\OperatorTok{!=}\StringTok{ }\DecValTok{1}\NormalTok{, ped}\OperatorTok{$}\NormalTok{Genotiped }\OperatorTok{!=}\StringTok{ }\DecValTok{1}\NormalTok{]}
  \CommentTok{# Individuos genotipados:}
\NormalTok{  A_}\DecValTok{22}\NormalTok{ <-}\StringTok{ }\NormalTok{Matrix_A[ped}\OperatorTok{$}\NormalTok{Genotiped }\OperatorTok{==}\StringTok{ }\DecValTok{1}\NormalTok{, ped}\OperatorTok{$}\NormalTok{Genotiped }\OperatorTok{==}\StringTok{ }\DecValTok{1}\NormalTok{]}
  \CommentTok{# Individuos no genotipados (en filas) y genotipados (en }
  \CommentTok{# columnas):}
\NormalTok{  A_}\DecValTok{12}\NormalTok{ <-}\StringTok{ }\NormalTok{Matrix_A[ped}\OperatorTok{$}\NormalTok{Genotiped }\OperatorTok{!=}\StringTok{ }\DecValTok{1}\NormalTok{, ped}\OperatorTok{$}\NormalTok{Genotiped }\OperatorTok{==}\StringTok{ }\DecValTok{1}\NormalTok{]}
  \CommentTok{# Transpuesta de la anterior (individuos no genotipados en }
  \CommentTok{# columnas y genotipados en filas):}
\NormalTok{  A_}\DecValTok{21}\NormalTok{ <-}\StringTok{ }\KeywordTok{t}\NormalTok{(A_}\DecValTok{12}\NormalTok{)}
  
  \CommentTok{# 3. Se coloca el nombre de las filas y y de las columnas }
  \CommentTok{# de la matriz G según los individuos genotipados}
  
  \KeywordTok{rownames}\NormalTok{(mG) <-}\StringTok{ }\NormalTok{ped}\OperatorTok{$}\NormalTok{id[ped}\OperatorTok{$}\NormalTok{Genotiped }\OperatorTok{==}\StringTok{ }\DecValTok{1}\NormalTok{]}
  \KeywordTok{colnames}\NormalTok{(mG) <-}\StringTok{ }\NormalTok{ped}\OperatorTok{$}\NormalTok{id[ped}\OperatorTok{$}\NormalTok{Genotiped }\OperatorTok{==}\StringTok{ }\DecValTok{1}\NormalTok{]}
  
  \CommentTok{# 4. Teniendo todos los componentes de la matriz H, se pro-}
  \CommentTok{# cede a su construcción}
  
\NormalTok{  H_}\DecValTok{11}\NormalTok{ <-}\StringTok{ }\NormalTok{A_}\DecValTok{11} \OperatorTok{-}\StringTok{ }
\StringTok{    }\NormalTok{(A_}\DecValTok{12} \OperatorTok\StringTok{ }\KeywordTok{solve}\NormalTok{(A_}\DecValTok{22}\NormalTok{) }\OperatorTok\StringTok{ }\NormalTok{A_}\DecValTok{21}\NormalTok{) }\OperatorTok{+}\StringTok{ }
\StringTok{    }\NormalTok{(A_}\DecValTok{12} \OperatorTok\StringTok{ }\KeywordTok{solve}\NormalTok{(A_}\DecValTok{22}\NormalTok{) }\OperatorTok\StringTok{ }\NormalTok{mG }\OperatorTok\StringTok{ }\KeywordTok{solve}\NormalTok{(A_}\DecValTok{22}\NormalTok{) }\OperatorTok\StringTok{ }\NormalTok{A_}\DecValTok{21}\NormalTok{)}
\NormalTok{  H_}\DecValTok{12}\NormalTok{ <-}\StringTok{ }\NormalTok{A_}\DecValTok{12} \OperatorTok\StringTok{ }\KeywordTok{solve}\NormalTok{(A_}\DecValTok{22}\NormalTok{) }\OperatorTok\StringTok{ }\NormalTok{mG}
\NormalTok{  H_}\DecValTok{21}\NormalTok{ <-}\StringTok{ }\KeywordTok{t}\NormalTok{(H_}\DecValTok{12}\NormalTok{)}
\NormalTok{  H_}\DecValTok{22}\NormalTok{ <-}\StringTok{ }\NormalTok{mG}
  
\NormalTok{  H_}\DecValTok{11}\NormalTok{_H_}\DecValTok{12}\NormalTok{ <-}\StringTok{ }\KeywordTok{cbind}\NormalTok{(H_}\DecValTok{11}\NormalTok{, H_}\DecValTok{12}\NormalTok{)}
\NormalTok{  H_}\DecValTok{21}\NormalTok{_H_}\DecValTok{22}\NormalTok{ <-}\StringTok{ }\KeywordTok{cbind}\NormalTok{(H_}\DecValTok{21}\NormalTok{, H_}\DecValTok{22}\NormalTok{)}
\NormalTok{  mH <-}\StringTok{ }\KeywordTok{rbind}\NormalTok{(H_}\DecValTok{11}\NormalTok{_H_}\DecValTok{12}\NormalTok{, H_}\DecValTok{21}\NormalTok{_H_}\DecValTok{22}\NormalTok{)}
  
\NormalTok{  mH <-}\StringTok{ }\NormalTok{mH[}\KeywordTok{order}\NormalTok{(}\KeywordTok{as.numeric}\NormalTok{(}\KeywordTok{rownames}\NormalTok{(mH))), }
           \KeywordTok{order}\NormalTok{(}\KeywordTok{as.numeric}\NormalTok{(}\KeywordTok{colnames}\NormalTok{(mH)))]}
\NormalTok{  mH <-}\StringTok{ }\KeywordTok{Matrix}\NormalTok{(mH)}
  
  \CommentTok{# 5. Finalmente se indica retornar la matriz H (mH)}
  
  \KeywordTok{return}\NormalTok{(mH)}
\NormalTok{  \}}
\end{Highlighting}
\end{Shaded}

\hypertarget{habilidad-predictiva}{%
\section{Habilidad predictiva}\label{habilidad-predictiva}}

\captionsetup[table]{labelformat=empty,skip=1pt}
\begin{longtable}{crrrr}
\caption*{
{\large Pedirí 1} \\ 
{\small 751 individuos en total}
} \\ 
\toprule
 & \multicolumn{2}{c}{Bayesiano} & \multicolumn{2}{c}{Penalizado} \\ 
 \cmidrule(lr){2-3} \cmidrule(lr){4-5}
Genotipados & Media & Desvío & Media & Desvíó \\ 
\midrule
\multicolumn{1}{l}{Densidad 0} \\ 
\midrule
0 & 0.515 & 0.061 & 0.513 & 0.055 \\ 
\midrule
\multicolumn{1}{l}{Densidad 1.000} \\ 
\midrule
148 & 0.483 & 0.051 & 0.464 & 0.062 \\ 
298 & 0.447 & 0.028 & 0.438 & 0.030 \\ 
451 & 0.493 & 0.002 & 0.495 & 0.000 \\ 
\midrule
\multicolumn{1}{l}{Densidad 10.000} \\ 
\midrule
148 & 0.490 & 0.054 & 0.479 & 0.063 \\ 
298 & 0.520 & 0.032 & 0.515 & 0.035 \\ 
451 & 0.586 & 0.002 & 0.585 & 0.000 \\ 
\midrule
\multicolumn{1}{l}{Densidad 100.000} \\ 
\midrule
148 & 0.496 & 0.056 & 0.482 & 0.064 \\ 
298 & 0.536 & 0.031 & 0.529 & 0.035 \\ 
451 & 0.581 & 0.002 & 0.580 & 0.000 \\ 
 \bottomrule
\end{longtable}
\captionsetup[table]{labelformat=empty,skip=1pt}
\begin{longtable}{crrrr}
\caption*{
{\large Pedirí 2} \\ 
{\small 1661 individuos en total}
} \\ 
\toprule
 & \multicolumn{2}{c}{Bayesiano} & \multicolumn{2}{c}{Penalizado} \\ 
 \cmidrule(lr){2-3} \cmidrule(lr){4-5}
Genotipados & Media & Desvío & Media & Desvíó \\ 
\midrule
\multicolumn{1}{l}{Densidad 0} \\ 
\midrule
0 & 0.506 & 0.058 & 0.515 & 0.051 \\ 
\midrule
\multicolumn{1}{l}{Densidad 1.000} \\ 
\midrule
148 & 0.479 & 0.060 & 0.477 & 0.065 \\ 
298 & 0.454 & 0.024 & 0.454 & 0.026 \\ 
451 & 0.490 & 0.003 & 0.495 & 0.000 \\ 
\midrule
\multicolumn{1}{l}{Densidad 10.000} \\ 
\midrule
148 & 0.483 & 0.067 & 0.479 & 0.069 \\ 
298 & 0.539 & 0.024 & 0.538 & 0.025 \\ 
451 & 0.584 & 0.003 & 0.585 & 0.000 \\ 
\midrule
\multicolumn{1}{l}{Densidad 100.000} \\ 
\midrule
148 & 0.483 & 0.067 & 0.482 & 0.070 \\ 
298 & 0.550 & 0.023 & 0.552 & 0.025 \\ 
451 & 0.579 & 0.002 & 0.580 & 0.000 \\ 
 \bottomrule
\end{longtable}
\captionsetup[table]{labelformat=empty,skip=1pt}
\begin{longtable}{crrrr}
\caption*{
{\large Pedirí 3} \\ 
{\small 2451 individuos en total}
} \\ 
\toprule
 & \multicolumn{2}{c}{Bayesiano} & \multicolumn{2}{c}{Penalizado} \\ 
 \cmidrule(lr){2-3} \cmidrule(lr){4-5}
Genotipados & Media & Desvío & Media & Desvíó \\ 
\midrule
\multicolumn{1}{l}{Densidad 0} \\ 
\midrule
0 & 0.495 & 0.059 & 0.509 & 0.054 \\ 
\midrule
\multicolumn{1}{l}{Densidad 1.000} \\ 
\midrule
148 & 0.484 & 0.051 & 0.480 & 0.058 \\ 
298 & 0.439 & 0.036 & 0.441 & 0.036 \\ 
451 & 0.490 & 0.002 & 0.495 & 0.000 \\ 
\midrule
\multicolumn{1}{l}{Densidad 10.000} \\ 
\midrule
148 & 0.491 & 0.055 & 0.488 & 0.057 \\ 
298 & 0.524 & 0.031 & 0.525 & 0.031 \\ 
451 & 0.583 & 0.005 & 0.585 & 0.000 \\ 
\midrule
\multicolumn{1}{l}{Densidad 100.000} \\ 
\midrule
148 & 0.494 & 0.055 & 0.492 & 0.058 \\ 
298 & 0.538 & 0.033 & 0.540 & 0.031 \\ 
451 & 0.578 & 0.002 & 0.580 & 0.000 \\ 
 \bottomrule
\end{longtable}

\hypertarget{habilidad-predictiva-1}{%
\section{Habilidad predictiva}\label{habilidad-predictiva-1}}

\captionsetup[table]{labelformat=empty,skip=1pt}
\begin{longtable}{lrrrr}
\caption*{
{\large Pedirí 1} \\ 
{\small 2061 individuos en total}
} \\ 
\toprule
 & \multicolumn{2}{c}{F2} & \multicolumn{2}{c}{F3} \\ 
 \cmidrule(lr){2-3} \cmidrule(lr){4-5}
Genotipados & Media & Desvío & Media & Desvío \\ 
\midrule
\multicolumn{1}{l}{Densidad 0} \\ 
\midrule
Ninguno & 0.577 & 0.092 & 0.562 & 0.086 \\ 
\midrule
\multicolumn{1}{l}{Densidad 1.000} \\ 
\midrule
F0-F1-F2-F3 & 0.627 & 0.060 & 0.610 & 0.064 \\ 
F1-F2 & 0.583 & 0.093 & 0.568 & 0.089 \\ 
F2 & 0.582 & 0.098 & 0.567 & 0.090 \\ 
\midrule
\multicolumn{1}{l}{Densidad 10.000} \\ 
\midrule
F0-F1-F2-F3 & 0.629 & 0.061 & 0.618 & 0.065 \\ 
F1-F2 & 0.583 & 0.093 & 0.569 & 0.088 \\ 
F2 & 0.581 & 0.097 & 0.568 & 0.090 \\ 
\midrule
\multicolumn{1}{l}{Densidad 100.000} \\ 
\midrule
F0-F1-F2-F3 & 0.629 & 0.061 & 0.618 & 0.065 \\ 
F1-F2 & 0.583 & 0.093 & 0.569 & 0.088 \\ 
F2 & 0.581 & 0.097 & 0.568 & 0.090 \\ 
 \bottomrule
\end{longtable}
\captionsetup[table]{labelformat=empty,skip=1pt}
\begin{longtable}{lrrrr}
\caption*{
{\large Pedirí 2} \\ 
{\small 2071 individuos en total}
} \\ 
\toprule
 & \multicolumn{2}{c}{F2} & \multicolumn{2}{c}{F3} \\ 
 \cmidrule(lr){2-3} \cmidrule(lr){4-5}
Genotipados & Media & Desvío & Media & Desvío \\ 
\midrule
\multicolumn{1}{l}{Densidad 0} \\ 
\midrule
Ninguno & 0.574 & 0.095 & 0.560 & 0.091 \\ 
\midrule
\multicolumn{1}{l}{Densidad 1.000} \\ 
\midrule
F0-F1-F2-F3 & 0.632 & 0.080 & 0.623 & 0.079 \\ 
F1-F2 & 0.590 & 0.092 & 0.577 & 0.092 \\ 
F2 & 0.590 & 0.092 & 0.576 & 0.092 \\ 
\midrule
\multicolumn{1}{l}{Densidad 10.000} \\ 
\midrule
F0-F1-F2-F3 & 0.635 & 0.079 & 0.627 & 0.081 \\ 
F1-F2 & 0.591 & 0.094 & 0.576 & 0.092 \\ 
F2 & 0.590 & 0.093 & 0.576 & 0.092 \\ 
\midrule
\multicolumn{1}{l}{Densidad 100.000} \\ 
\midrule
F0-F1-F2-F3 & 0.635 & 0.079 & 0.627 & 0.081 \\ 
F1-F2 & 0.591 & 0.094 & 0.576 & 0.092 \\ 
F2 & 0.590 & 0.093 & 0.576 & 0.092 \\ 
 \bottomrule
\end{longtable}
\captionsetup[table]{labelformat=empty,skip=1pt}
\begin{longtable}{lrrrr}
\caption*{
{\large Pedirí 3} \\ 
{\small 2091 individuos en total}
} \\ 
\toprule
 & \multicolumn{2}{c}{F2} & \multicolumn{2}{c}{F3} \\ 
 \cmidrule(lr){2-3} \cmidrule(lr){4-5}
Genotipados & Media & Desvío & Media & Desvío \\ 
\midrule
\multicolumn{1}{l}{Densidad 0} \\ 
\midrule
Ninguno & 0.582 & 0.053 & 0.558 & 0.042 \\ 
\midrule
\multicolumn{1}{l}{Densidad 1.000} \\ 
\midrule
F0-F1-F2-F3 & 0.646 & 0.039 & 0.626 & 0.035 \\ 
F1-F2 & 0.607 & 0.053 & 0.583 & 0.048 \\ 
F2 & 0.605 & 0.053 & 0.579 & 0.049 \\ 
\midrule
\multicolumn{1}{l}{Densidad 10.000} \\ 
\midrule
F0-F1-F2-F3 & 0.652 & 0.040 & 0.632 & 0.031 \\ 
F1-F2 & 0.607 & 0.052 & 0.583 & 0.048 \\ 
F2 & 0.607 & 0.052 & 0.581 & 0.049 \\ 
\midrule
\multicolumn{1}{l}{Densidad 100.000} \\ 
\midrule
F0-F1-F2-F3 & 0.652 & 0.040 & 0.632 & 0.031 \\ 
F1-F2 & 0.607 & 0.052 & 0.583 & 0.048 \\ 
F2 & 0.607 & 0.052 & 0.581 & 0.049 \\ 
 \bottomrule
\end{longtable}
\captionsetup[table]{labelformat=empty,skip=1pt}
\begin{longtable}{lrrrr}
\caption*{
{\large Pedirí 4} \\ 
{\small 2131 individuos en total}
} \\ 
\toprule
 & \multicolumn{2}{c}{F2} & \multicolumn{2}{c}{F3} \\ 
 \cmidrule(lr){2-3} \cmidrule(lr){4-5}
Genotipados & Media & Desvío & Media & Desvío \\ 
\midrule
\multicolumn{1}{l}{Densidad 0} \\ 
\midrule
Ninguno & 0.556 & 0.047 & 0.538 & 0.044 \\ 
\midrule
\multicolumn{1}{l}{Densidad 1.000} \\ 
\midrule
F0-F1-F2-F3 & 0.624 & 0.036 & 0.609 & 0.038 \\ 
F1-F2 & 0.586 & 0.049 & 0.570 & 0.049 \\ 
F2 & 0.586 & 0.047 & 0.568 & 0.050 \\ 
\midrule
\multicolumn{1}{l}{Densidad 10.000} \\ 
\midrule
F0-F1-F2-F3 & 0.628 & 0.034 & 0.615 & 0.038 \\ 
F1-F2 & 0.586 & 0.050 & 0.571 & 0.050 \\ 
F2 & 0.585 & 0.050 & 0.569 & 0.051 \\ 
\midrule
\multicolumn{1}{l}{Densidad 100.000} \\ 
\midrule
F0-F1-F2-F3 & 0.628 & 0.034 & 0.615 & 0.038 \\ 
F1-F2 & 0.586 & 0.050 & 0.571 & 0.050 \\ 
F2 & 0.585 & 0.050 & 0.569 & 0.051 \\ 
 \bottomrule
\end{longtable}

\backmatter

\hypertarget{bibliografuxeda}{%
\chapter*{Bibliografía}\label{bibliografuxeda}}
\addcontentsline{toc}{chapter}{Bibliografía}

\markboth{\MakeUppercase{Bibliography}}{} % have to explicitly state what to put in the heading (bug in bookdown?)
%format the references so they have a hanging indent. Remove these (and the \endgroup command) if you want regular indentation.
\begingroup
\hspace{\parindent}
\setlength{\parindent}{-0.25in}
\setlength{\leftskip}{0.25in}
\setlength{\parskip}{0pt}

\hypertarget{refs}{}
\leavevmode\hypertarget{ref-cite:33}{}%
Ahmadi, N., J. Bartholomé, T. V. Cao, y C. Grenier. 2020. \emph{Quantitative genetics, genomics and plant breeding}. 2nd edition. \url{https://doi.org/10.1079/9781789240214.0243}.

\leavevmode\hypertarget{ref-cite:21}{}%
Blasco, A. 2021. \emph{Mejora genética animal}. 1st edition. EDITORIAL SÍNTESIS, S. A.

\leavevmode\hypertarget{ref-cite:3}{}%
Blasco, A., y M. A. Toro. 2014. «A short critical history of the application of genomics to animal breeding». \emph{Livestock Science} 166: 4-9.

\leavevmode\hypertarget{ref-cite:42}{}%
Caligari, P. D. S., y J. Brown. 2017. \emph{Plant breeding, practice}. 2nd edition. Vol. 2. Elsevier Ltd. \url{https://doi.org/10.1016/B978-0-12-394807-6.00195-7}.

\leavevmode\hypertarget{ref-cite:37}{}%
Crossa, J., P. Pérez-Rodríguez, J. Cuevas, O. Montesinos-López, D. Jarquín, G. delosCampos, J. Burgueño, et~al. 2017. «Genomic selection in plant breeding: methods, models, and perspectives». \emph{Trends in Plant Science}, 961-75. \url{https://doi.org/10.1016/j.tplants.2017.08.011}.

\leavevmode\hypertarget{ref-cite:31}{}%
delosCampos, G., J. H. Hickey, R. Pong-Wong, H. D. Daetwyler, y M. P. L. Calus. 2013. «Whole-genome regression and prediction methods applied to plant and animal breeding». \emph{Genetics} 193: 327-45. \url{https://doi.org/10.1534/genetics.112.143313}.

\leavevmode\hypertarget{ref-cite:10}{}%
Desta, Z. A., y R. Ortiz. 2014. «Genomic selection: genome-wide prediction in plant improvement». \emph{Trends in Plant Science} 19 (9): 592-601.

\leavevmode\hypertarget{ref-cite:24}{}%
Fernández, J., y M. Toro. 2006. «A new method to estimate relatedness from molecular markers». \emph{Molecular Ecology} 15: 1657-67.

\leavevmode\hypertarget{ref-cite:1}{}%
Fisher, R. A. 1918. «The correlaction between relatives under the supposition of Mendelian inheritance». \emph{Transactions of the Royal Society of Edinburgh} 52: 399-433.

\leavevmode\hypertarget{ref-cite:28}{}%
Freeman, A. E. 1991. «C. R. Henderson: contributions to the dairy industry». \emph{Journal of Dairy Science} 74 (11): 4045-51. \url{https://doi.org/10.3168/jds.S0022-0302(91)78600-1}.

\leavevmode\hypertarget{ref-cite:35}{}%
Grinberg, N. F., O. I. Orhobor, y R. D. King. 2020. «An evaluation of machine-learning for predicting phenotype: studies in yeast, rice, and wheat». \emph{Machine Learning} 109: 251-77. \url{https://doi.org/10.1007/s10994-019-05848-5}.

\leavevmode\hypertarget{ref-cite:13}{}%
Hayes, B. J., P. M. Visscher, y M. E. Goddard. 2009. «Increased accuracy of artificial selection by using the realized relationship matrix». \emph{Genetics Research} 91: 47-60.

\leavevmode\hypertarget{ref-cite:41}{}%
Henderson, C. R. 1975. «Best linear unbiased estimation and prediction under a selection model». \emph{Biometrics} 31: 423-47.

\leavevmode\hypertarget{ref-cite:44}{}%
Hickey, J. M., T. Chiurugwi, I. Mackay, W. Powell, y Implementing Genomic Selection in CGIAR Breeding Programs Workshop Participants. 2017. «Genomic prediction unifies animal and plant breeding programs to form platforms for biological discovery». \emph{Nature Genetics} 49 (9): 1297-1303. \url{https://doi.org/10.1038/ng.3920}.

\leavevmode\hypertarget{ref-cite:43}{}%
Holland, J. B. 2014. \emph{Breeding: plants, modern}. Vol. 2. Elsevier Inc. \url{https://doi.org/10.1016/B978-0-444-52512-3.00226-6}.

\leavevmode\hypertarget{ref-cite:20}{}%
Imai, A., T. Kuniga, T. Yoshioka, K. Nonaka, N. Mitani, H. Fukamachi, N. Hiehata, M. Yamamoto, y T. Hayashi. 2019. «Single-step genomic prediction of fruit-quality traits using phenotypic records of non-genotyped relatives in citrus». \emph{PLoS ONE} 14 (8). \url{https://doi.org/10.1371/journal.pone.0221880}.

\leavevmode\hypertarget{ref-cite:11}{}%
Jurcic, E. J., P. V. Villalba, P. S. Pathauer, D. A. Palazzini, G. P. J. Oberschelp, L. Harrand, M. N. Garcia, et~al. 2021. «Single-setp genomic prediction of Eucalyptus dunni using different identity-by-descent and identity-by-state relationship matrices». \emph{Heredity} 127: 176-89.

\leavevmode\hypertarget{ref-cite:32}{}%
Kyselova, J., L. Tichý, y K. Jochová. 2021. «The role of molecular genetics in animal breeding: a minireview». \emph{Czech Journal of Animal Science} 66 (4): 107-11. \url{https://doi.org/10.17221/251/2020-CJAS}.

\leavevmode\hypertarget{ref-cite:17}{}%
Legarra, A., I. Aguilar, y I. Misztal. 2009. «A relationship matrix including full pedigree and genomic information». \emph{Journal of Dairy Science} 92: 4656-63. \url{https://doi.org/10.3168/jds.2009-2061}.

\leavevmode\hypertarget{ref-cite:15}{}%
Legarra, A., O. F. Christensen, I. Aguilar, y I. Misztal. 2014. «Single Step, a general approach for genomic selection». \emph{Livestock Science}. \url{https://doi.org/10.1016/j.livsci.2014.04.029}.

\leavevmode\hypertarget{ref-cite:30}{}%
Legarra, A., D. Lourenco, y Z. G. Vitezica. 2018. \emph{Bases for genomic prediction}.

\leavevmode\hypertarget{ref-cite:22}{}%
Lourenco, D., A. Legarra, S. Tsuruta, Y. Masuda, I. Aguilar, y I. Misztal. 2020. «Single-step genomic evaluations from theory to practice: using SNP chips and sequence data in BLUPF90». \emph{Genes} 11: 790. \url{https://doi.org/doi:10.3390/genes11070790}.

\leavevmode\hypertarget{ref-cite:38}{}%
Medina, C. A., H. Kaur, I. Ray, y L. X. Yu. 2021. «Strategies to Increase Prediction Accuracy in Genomic Selection of Complex Traits in Alfalfa (Medicago sativa L.)». \emph{Cells} 10 (12). \url{https://doi.org/10.3390/cells10123372}.

\leavevmode\hypertarget{ref-cite:8}{}%
Meuwissen, T. H. E., B. J. Hayes, y M. E. Goddard. 2001. «Prediction of Total Genetic Value Using Genome-Wide Dense Marker Maps». \emph{Genetics} 157: 1819-29.

\leavevmode\hypertarget{ref-cite:14}{}%
Misztal, I., S. E. Aggrrey, y W. M. Muir. 2012. «Experiences with a single-step genome evaluation». \emph{Poultry Science} 92: 2530-4.

\leavevmode\hypertarget{ref-cite:16}{}%
Misztal, I., A. Legarra, y I. Aguilar. 2009. «Computing procedures for genetic evaluation including phenotypic, full pedigree, and genomic information». \emph{Journal of Dairy Science} 92: 4648-55. \url{https://doi.org/10.3168/jds.2009-2064}.

\leavevmode\hypertarget{ref-cite:18}{}%
Misztal, I., D. Lourenco, y A. Legarra. 2020. «Current status of genomic evaluation». \emph{Journal of Animal Science} 98 (4): 1-14. \url{https://doi.org/10.1093/jas/skaa101}.

\leavevmode\hypertarget{ref-cite:6}{}%
Nakaya, A., y S. N. Isobe. 2012. «Will genomic selection be a practical method for plant breeding?» \emph{Annals of Botany} 110: 1303-16.

\leavevmode\hypertarget{ref-cite:2}{}%
Nelson, R. M., M. E. Pettersson, y Ö. Carlborg. 2012. «A century after Fisher: time for a new paradigm in quantitative genetics». \emph{Trends in Genetics} 29 (9): 669-76.

\leavevmode\hypertarget{ref-cite:5}{}%
Pérez-Enciso, M. 2017. «Animal breeding learning from machine learning». \emph{Journal of Animal Breeding and Genetics} 134: 85-86.

\leavevmode\hypertarget{ref-cite:19}{}%
Pérez-Rodríguez, P., J. Crossa, J. Rutkoski, J. Poland, R. Singh, A. Legarra, E. Autrique, J. Burgueño G. delosCampos, y S. Dreisigacker. 2017. «Single-step genomic and pedigree genotype x environment interaction models for predicting wheat lines in international environments». \emph{Plant Genome} 10 (2). \url{https://doi.org/10.3835/plantgenome2016.09.0089}.

\leavevmode\hypertarget{ref-cite:36}{}%
Qanbari, S. 2020. «On the extent of linkage disequilibrium in the genome of farm animals». \emph{Frontiers in Genetics} 10. \url{https://doi.org/10.3389/fgene.2019.01304}.

\leavevmode\hypertarget{ref-cite:27}{}%
Schaeffer, L. R. 1991. «C. R. Henderson: contributions to predicting genetic merit». \emph{Journal of Dairy Science} 74 (11): 4052-66. \url{https://doi.org/10.3168/jds.S0022-0302(91)78601-3}.

\leavevmode\hypertarget{ref-cite:29}{}%
Searle, S. R. 1991. «C. R. Henderson, the statistician; and his contributions to variance components estimation». \emph{Journal of Dairy Science} 74 (11): 4035-44. \url{https://doi.org/10.3168/jds.S0022-0302(91)78599-8}.

\leavevmode\hypertarget{ref-cite:34}{}%
Tan, C., C. Bian, D. Yang, N. Li, Z. Wu, y X. Hu. 2017. «Application of genomic selection in farm animal breeding». \emph{Hereditas} 39 (11): 1033-45. \url{https://doi.org/10.16288/j.yczz.17-286}.

\leavevmode\hypertarget{ref-cite:7}{}%
Tong, H., y Z. Nikoloski. 2021. «Machine learning approaches for crop improvement: leveraging phenotypic and genotypic big data». \emph{Journal of Plant Physiology} 257: 153354. \url{https://doi.org/10.1016/j.jplph.2020.153354}.

\leavevmode\hypertarget{ref-cite:9}{}%
Turelli, M. 2017. «Prediction of Total Genetic Value Using Genome-Wide Dense Marker Maps». \emph{Theoretical Population Biology} 118: 46-49.

\leavevmode\hypertarget{ref-cite:39}{}%
VanRaden, P. M. 2007. «Efficient methods to compute genomic predictions». \emph{Journal of Dairy Science} 91: 4414-23.

\leavevmode\hypertarget{ref-cite:4}{}%
Villemereuil, P. de, H. Schielzeth, S. Nakagawa, y M. Morrissey. 2016. «General methods for evolutionary quantitative genetic inference from generalized mixed models». \emph{Genetics} 204: 1281-94.

\leavevmode\hypertarget{ref-cite:26}{}%
Vourlaki, I., R. Castanera, S. Ramos-Onsins, J. Casacuberta, y M. Pérez-Enciso. s.~f. «Transposable element polymorphisms improve prediction of complex agronomic traits in rice». \emph{Frontiers in Plant Science}.

\leavevmode\hypertarget{ref-cite:46}{}%
Wang, J., J. Crossa, y J. Gai. 2020. «Quantitative genetic studies with applications in plant breeding in the omics era». \emph{The Crop Journal} 8: 683-87. \url{https://doi.org/10.1016/j.cj.2020.09.001}.

\leavevmode\hypertarget{ref-cite:12}{}%
Wright, S. 1922. «Coefficients of inbreeding and relationship». \emph{The American Naturalist} 56: 330-38.

\leavevmode\hypertarget{ref-cite:25}{}%
Xua, S., D. Zhub, y Q. Zhang. 2014. «Predicting hybrid performance in rice using genomic best linear unbiased prediction». \emph{Proceedings of the National Academy of Sciences of the United States of America} 111 (34): 12456-61. \url{https://doi.org/10.1073/pnas.1413750111}.

\endgroup

\hypertarget{agradecimientos}{%
\chapter*{Agradecimientos}\label{agradecimientos}}
\addcontentsline{toc}{chapter}{Agradecimientos}

\chaptermark{Acknowledgments}

\includegraphics{figures/uvalogo_regular_p_en.pdf}

\backmatter

\end{document}
